\documentclass[specialist,
  substylefile = spbu.rtx,
  href,
  colorlinks=true,
  12pt]{disser}

\usepackage[utf8]{inputenc}
\usepackage[T2A]{fontenc}
\usepackage[english, russian]{babel}
\usepackage[a4paper,
  mag=1000,
  includefoot,
  left=3cm,
  right=1.5cm,
  top=2cm,
  bottom=2cm,
  headsep=1cm,
  footskip=1cm]{geometry}
\usepackage{longtable}
\usepackage{subcaption}
\usepackage{fancyvrb}
\usepackage{float}
\usepackage{listings}
\lstset{extendedchars=\true}
\usepackage{xcolor}
\usepackage{amsmath, amsfonts, amsthm}
\usepackage{graphics}
\usepackage{csquotes}
\usepackage{nicefrac}
\usepackage{minted}
\usepackage[
  backend=biber,
  bibstyle=gost-numeric,
  citestyle=gost-numeric,
  hyperref=true,
  natbib=true,
  url=true,
]{biblatex}

\usepackage{mathtools}
\usepackage{romannum}
\hypersetup{unicode=true}
\addbibresource{./ref.bib}
\let\vec=\mathbf
\setcounter{tocdepth}{2}
\author{Илья Шешуков}
\date{\number\year}
\title{Отчёт по научно-исследовательской работе}
\hypersetup{
 pdfauthor={Илья Шешуков},
 pdftitle={Курсовая работа},
 pdfkeywords={},
 pdfsubject={},
 pdfcreator={Emacs 26.3},
 pdflang={Russian}}

\newtheorem{utv}{Утверждение}

% Имена без курсива
\renewcommand*{\mkgostheading}[1]{#1}
% Правка записей типа thesis, чтобы дважды не писался автор
\DeclareBibliographyDriver{thesis}{
  \usebibmacro{bibindex}%
  \usebibmacro{begentry}%
  \usebibmacro{heading}%
  \newunit
  \usebibmacro{author}%
  \setunit*{\labelnamepunct}%
  \usebibmacro{thesistitle}%
  \setunit{\respdelim}%
  %\printnames[last-first:full]{author}%Вот эту строчку нужно убрать, чтобы автор диссертации не дублировался
  \newunit\newblock
  \printlist[semicolondelim]{specdata}%
  \newunit
  \usebibmacro{institution+location+date}%
  \newunit\newblock
  \usebibmacro{chapter+pages}%
  \newunit
  \printfield{pagetotal}%
  \newunit\newblock
  \usebibmacro{doi+eprint+url+note}%
  \newunit\newblock
  \usebibmacro{addendum+pubstate}%
  \setunit{\bibpagerefpunct}\newblock
  \usebibmacro{pageref}%
  \newunit\newblock
  \usebibmacro{related:init}%
  \usebibmacro{related}%
  \usebibmacro{finentry}
}

\theoremstyle{definition}
\newtheorem{definition}{Определение}[section]


\institution{
    Санкт-Петербургский государственный университет \\
    Прикладная математика и информатика \\
    Вычислительная стохастика и статистические модели
}

\topic{\normalfont\scshape%
Разработка вычислительных алгоритмов и программных средств поддержки принятия решений
}

% Научный руководитель
\sa       {Н.\,К.~Кривулин}
\sastatus {доктор физ.-мат. наук, профессор}

% Город и год
\city{Санкт-Петербург}


\begin{document}

\maketitle
\addcontentsline{toc}{chapter}{\nameref{sec:intro}}
\tableofcontents

\chapter*{Введение}
\label{sec:intro}
Задача принятия решения на основе парных сравнений состоит в том, чтобы по результатам попарных сравнений $n$ альтернатив определить абсолютный приоритет каждой альтернативы.

\begin{definition}{Матрица парных сравнений}
  есть матрица $A = (a_{ij})$, элемент $a_{ij}$ которой показывает, во сколько раз альтернатива $i$ превосходит альтернативу $j$ и для которой верны условия
  \[
    a_{ij} = \nicefrac{1}{a_{ij}}, \quad a_{ij} > 0.
  \]
\end{definition}
\begin{definition}{Согласованная матрица}
  есть матрица $A = (a_{ij})$ такая, что выполняется свойство транзитивности. Т.е.
  \[
    a_{ij} = a_{ik}a_{kj}, \quad i,j,k = 1,\dots,n.
  \]
\end{definition}
\begin{definition}
  В том случае, если $A$ согласованна, то существует вектор $x$, называемый \textit{вектором абсолютных приоритетов}, такой, что
  \[
    a_{ij} = x_{i}/x_{j}.
  \]
\end{definition}
   На практике, матрицы парных сравнений обычно несогласованны, и поэтому возникает задача приближения матрицы парных сравнений согласованной матрицей.
  Одним из способов решения (т.н. $\log$-чебышёвская аппроксимация) этой задачи является нахождение вектора $x$, который бы минимизировал следующее выражение
  \[
  l_{\infty}\left(A, x x^{-}\right)=\max _{1 \leq i, j \leq n}\left|\log a_{i j}-\log \frac{x_{i}}{x_{j}}\right|, \quad x^{-} = (x^{T})^{-1},
  \]
  что сводится к нахождению $x$ такого, что
  \[
    \min _{x>0} \max _{1 \leq i, j \leq n} \frac{a_{i j} x_{j}}{x_{i}}.
  \]
  Данная задача может быть решена с использованием идемпотентной алгебры и max-алгебры в частности.
\begin{definition}{Max-алгебра}
  есть алгебра над множеством $\mathbb{R}_{+}=\{x \in \mathbb{R} \mid x \geq 0\}$ с операциями ${(\oplus, \times)}$, где $\oplus$ --- это максимум, а $\times$ --- стандартное умножение.
\end{definition}
Соответственно, на языке max-алгебры решение задачи будет выглядеть следующим образом
\[
  \min _{x>0} \bigoplus_{1 \leq i, j \leq n} x_{i}^{-1} a_{i j} x_{j},
\]
что можно дополнительно переписать как
\[
  \min _{x>0} x^{-} A x.
\]

\section{Задача}
  Требуется написать программу, символьно решающую задачу принятия решений методом log-чебышёвской аппроксимации в max-алгебре.

\chapter{Ход работы}
\section{Выполнено}
  \begin{enumerate}
    \item Было написано расширение для библиотеки линейной алгебры Eigen, позволяющее работать с символьными вычислениями в max-алгебре.
    \item Были реализованы некоторые функции, используемые в алгоритме решения задачи: функция вычисления спектрального радиуса и оператор Клини.
  \end{enumerate}

\chapter{Заключение}
\section{Планируется сделать}
  \begin{enumerate}
    \item Добавить больше функций
    \item Добавить в программу возможность решения многокритериальных задач
    \item Сравнить скорость и результаты работы программы с существующими решениями
  \end{enumerate}

\appendix
\cleardoublepage\makeatletter\@openrightfalse\makeatother
\chapter{Примеры работы программы}
\section{Пример 1}
Объявление двух матриц $A_{1}$ и $A_{2}$ размера $3\times 3$:
  \begin{lstlisting}[language=C++,basicstyle=\ttfamily,keywordstyle=\color{red}]
 MatrixDn A1(3, 3);
 MatrixDn A2(3, 3);

 A1 << ex(sin(1)), 3, ex(1) / 3,
        ex(1) / 3, 1, 1,
                3, 1, 1;
 A2 <<   0, ex(1) / 3, 5,
         3,         1, 7,
 ex(1) / 5, ex(1) / 7, 1;
  \end{lstlisting}
  Сумма и произведение этих матриц:
  \begin{lstlisting}[language=C++,basicstyle=\ttfamily,keywordstyle=\color{red}]
 std::cout << "A1+A2=\n" << A1+A2 << std::endl;
 std::cout << "A1*A2=\n" << A1*A2 << std::endl;
  \end{lstlisting}
  Вывод:
  \begin{lstlisting}[language=bash,basicstyle=\ttfamily,keywordstyle=\color{red}]
 A1 + A2 =               A1 * A2 =
   \sin(1) 3 5             9  3 21
   3       1 7             3  1  7
   3       1 1             3  1 15
  \end{lstlisting}
\section{Пример 2}
Объявление матрицы $C$ размера $4\times 4$:
  \begin{lstlisting}[language=bash,basicstyle=\ttfamily,keywordstyle=\color{red}]
   MatrixDn C(4, 4);
   C <<  8, 3, 10, 6,
         7, 2,  1, 4,
        10, 8,  2, 2,
         4, 7,  6, 1;
  \end{lstlisting}
  Вычисление спектрального радиуса:
  \begin{lstlisting}[language=bash,basicstyle=\ttfamily,keywordstyle=\color{red}]
   std::cout
         << "Spectral radius of C = "
         << spectral_radius(C)
         << std::endl;
   \end{lstlisting}
Вывод:
  \begin{lstlisting}[language=bash,basicstyle=\ttfamily,keywordstyle=\color{red}]
m^1=                      m^2=
 8  3 10  6               100  80  80  48
 7  2  1  4                56  28  70  42
10  8  2  2                80  30 100  60
 4  7  6  1                60  48  40  28
                          tr(m^2)^(1/2) = 10

m^3=                      m^4=
 800  640 1000  600       10000  8000  8000  4800
 700  560  560  336        5600  4480  7000  4200
1000  800  800  480        8000  6400 10000  6000
 480  320  600  360        6000  4800  4800  2880
tr(m^3)^(1/3) = 800^(1/3)   tr(m^4)^(1/4) = 10

Spectral radius of C = 10
\end{lstlisting}
\end{document}
