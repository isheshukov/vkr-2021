\documentclass[ucs, notheorems, handout]{beamer}

\usetheme[numbers,totalnumbers,compress, nologo]{Statmod}
\usefonttheme[onlymath]{serif}
\setbeamertemplate{navigation symbols}{}

\mode<handout> {
    \usepackage{pgfpages}
    \setbeameroption{show notes}
    \pgfpagesuselayout{2 on 1}[a4paper, border shrink=5mm]
}

\usepackage[utf8x]{inputenc}
\usepackage[T2A]{fontenc}
\usepackage[russian]{babel}
\usepackage{tikz}
\usepackage{ragged2e}
\usepackage{nicefrac}
\usepackage{listings}

\makeatletter
\defbeamertemplate*{footline}{statmod theme mod}
{%
    \leavevmode%
    \hbox{%
        \begin{beamercolorbox}[wd=.4\paperwidth,ht=2.5ex,dp=1.125ex,leftskip=.3cm,rightskip=.3cm]{author in head/foot}%
            \usebeamerfont{author in head/foot}\insertframenumber{}%
            \ifbeamer@totalnumbers
            /\inserttotalframenumber
            \fi
            \hfill\insertshortauthor
        \end{beamercolorbox}%
        \begin{beamercolorbox}[wd=.6\paperwidth,ht=2.5ex,dp=1.125ex,leftskip=.3cm,rightskip=.3cm plus1fil]{title in head/foot}%
            \usebeamerfont{title in head/foot}\insertshorttitle
        \end{beamercolorbox}}%
    \vskip0pt%
}

\newtheorem{theorem}{Теорема}

\title[Разработка алгоритмов и ПО поддержки принятия решений]{%
	Разработка вычислительных алгоритмов и программных средств поддержки принятия решений}

\author{Шешуков Илья Вячеславович}

\institute[Санкт-Петербургский Государственный Университет]{%
    \small
    Санкт-Петербургский государственный университет\\
    Прикладная математика и информатика\\
    Вычислительная стохастика и статистические модели\\
    \vspace{1.25cm}
    Преддипломная практика}

\date[Защита]{Санкт-Петербург, 2021}

\subject{Talks}

\begin{document}

\begin{frame}[plain]
    \titlepage

    \note{Работа выполнена на кафедре статистического моделирования,\\
	руководитель доктор физ.-мат. наук, профессор Н.\,К.~Кривулин}
\end{frame}


\section{Характеристика работы}
\subsection{Введение}

\setbeameroption{show notes}

\begin{frame}
    \begin{itemize}
      \item Задача принятия решения на основе парных сравнений состоит в том, чтобы по результатам попарных сравнений $n$ альтернатив определить абсолютный приоритет каждой альтернативы. 
      \item Подобные задачи возникают в социологии, маркетинге, менеджменте.
      \item Одним из методов решения подобных задач является метод log-чебышёвской аппроксимации. 
      \item Существуют и другие методы: метод Саати, геометрических средних.
      \item Преимущество log-чебышёвской аппроксимации в том, что решение получют аналитически.
    \end{itemize}

    \note{
	\tiny Задача принятия решения на основе парных сравнений состоит в том, чтобы по результатам попарных сравнений $n$ альтернатив определить абсолютный приоритет каждой альтернативы. 
	Подобные задачи возникают во множестве областей: социологии, маркетинге, менеджменте.
	Например, человеку или бизнесу нужно сделать выбор из нескольких альтернатив. 
	Он или эксперт сранивают каждую пару альтернатив между собой (это для человека проще) и алгоритм выстроит все альтернативы по степени их предпочтительности.
	Работа была начата в этом году, в ходе работы были изучены материалы на тему тро­пической оптимизации и было начато написание библиотеки для решения задачи принятия решений.
	
	\tiny Одним из методов решения подобных задач является метод log-чебышёвской аппроксимации. 
	Существуют и другие методы: метод Саати, метод геометрических средних, однако преимущество
	метода log-чебышёвской аппроксимации в том, что решения получают в аналитическом виде, а не в численном. 
	Однако, пока не существует общедоступных программных средств, которые бы реализовывали этот метод и решали данную задачу 
	на компьютере.
    }
\end{frame}

\section{Постановка задачи}
\subsection{Важные определения}
\begin{frame}
	\begin{block}{Матрица парных сравнений}
		Матрицей парных сравнений называется матрица $\boldsymbol{A} = (a_{ij})$, элемент $a_{ij}$ которой показывает, во сколько раз альтернатива $i$ превосходит $j$ и верно
		\[
		a_{ij} = \nicefrac{1}{a_{ji}}, \quad a_{ij} > 0.
		\]
	\end{block}
	\begin{block}{Согласованная матрица}
		Матрица называется согласованной, если выполняется условие транзитивности:
		$$
		a_{i j}=a_{i k} a_{k j}, \quad i,j,k = 1. \dots, n.
		$$
	\end{block}
	\note{
		Для описания задачи нужно привести несколько определений.
		Матрицей парных сравнений называется матрица $\boldsymbol{A} = (a_{ij})$, элемент $a_{ij}$ которой показывает, во сколько раз альтернатива $i$ превосходит $j$ и верно условие.
		Согласованной называется матрица, если выполняется условие транзитивности:
	}
\end{frame}

\subsection{Свойство}
\begin{frame}
	\begin{itemize}
	\begin{block}{Вектор альтернатив}
	В случае, если матрица парных сравнений $\mathbf{A}$ является согласованной, то выполняется [Saaty, 1984]:
	$$
		\exists \mathbf{x} > 0: a_{ij} = x_{i}/x_{j} \quad \forall i, j.
	$$
	\end{block}
	\item На практике, матрицы парных альтернатив редко являются согласованными.
	\item Возникает задача аппроксимации $\mathbf{A}$ некоторой согласованной матрицей.
	\end{itemize}
	\note{
		Важным является следующий факт. 
		И если матрица парных сравнений $\mathbf{A}$ является согласованной, то выполняется условие.
		Собственно, для согласованных матриц парных сравнений этот вектор и есть решение.
		Но в реальной жизни, матрицы парных сравнений такими обычно не является, поэтому возникает
		задача их аппроксимации некоторой согласованной.
	}
\end{frame}
\subsection{log-чебышёвская аппроксимация}
\begin{frame}
	Рассматриваемым здесь способом решения [Кривулин, 2019] этой задачи является нахождение вектора $\mathbf{x}$, который бы минимизировал следующее выражение
	\[
	l_{\infty}\left(\mathbf{A}, \mathbf{x} \mathbf{x}^{-}\right)=\max _{1 \leq i, j \leq n}\left|\log a_{i j}-\log \frac{x_{i}}{x_{j}}\right|, \quad \mathbf{x}^{-} = \begin{pmatrix}x_1^{-1}&x_2^{-1} & \dots & x_n^{-1}\end{pmatrix},
	\]
	что сводится к решению задачи
	\[
	\min _{\mathbf{x}>\mathbf{0}} \max _{1 \leq i, j \leq n} \frac{a_{i j} x_{j}}{x_{i}}.
	\]
	Данная задача может быть решена с использованием идемпотентной алгебры и max-алгебры в частности.
	\note{
		Важным является следующий факт. 
		И если матрица парных сравнений $\mathbf{A}$ является согласованной, то выполняется условие.
		Собственно, для согласованных матриц парных сравнений этот вектор и есть решение.
		Но в реальной жизни, матрицы парных сравнений такими обычно не является, поэтому возникает
		задача их аппроксимации некоторой согласованной.
	}
\end{frame}

\subsection{Идемпотентное представление задачи}
\begin{frame}
	 \begin{block}{Max-$\times$-алгебра}
		$\text{Max-}\times$-алгеброй называется алгебра над множеством $\mathbb{R}_{+}=\{x \in \mathbb{R} \mid x \geq 0\}$ с операциями ${(\oplus, \times)}$, где $\oplus$ --- это максимум, а $\times$ --- стандартное умножение.
	\end{block}
	
	\begin{block}{Однокритериальная задача парного сравнения}
		Соответственно, задача будет выглядеть следующим образом
		\begin{align*}
		\min _{\mathbf{x}>\mathbf{0}} \max _{1 \leq i, j \leq n} \frac{a_{i j} x_{j}}{x_{i}} && \rightarrowtail &&
		\min _{\mathbf{x}>\mathbf{0}} \bigoplus_{1 \leq i, j \leq n} x_{i}^{-1} a_{i j} x_{j},
		\end{align*}
		что можно дополнительно переписать в векторном виде как
		\[
		\min _{\mathbf{x}>\mathbf{0}} \mathbf{x}^{-} \mathbf{A} \mathbf{x}.
		\]
	\end{block}

	\note{
		Max-* алгебра это алгебра, у которой вместо операции сложение используют максимум.
		Её можно естественным образом расширить на векторные операции.
		И в новой терминологии задача приобретает следующий вид.
		Для решения последней задачи есть дополнительная теория, которую здесь приводить не будем.
	}
\end{frame}

\section{Работа}
\begin{frame}{}
	\begin{block}{Цель работы}
		Написать библиотеку и программу на языке \texttt{C++}, символьно решающую задачу принятия решений методом log-чебышёвской аппроксимации в max-алгебре.
	\end{block}
	
	\begin{block}{Результаты}
		\begin{enumerate}
			\item Были релизованы max-+ и max-$\times$-алгебры и добавлена их поддержка в библиотеку линейной алгебры \texttt{Eigen} (версия 3.3.8).
			\item Все вычисление проводятся символьно (при помощи библиотеки \texttt{GiNaC} версии 1.8.0).
			\item Был реализован набор функций, используемых в алгоритме решения задачи.
			\item Многокритериальная задача принятия решений может быть решена полностью при помощи программы.
		\end{enumerate}
	\end{block}
	
	\note{
		\tiny Цель работы: написать библиотеку и программу на языке \texttt{C++}, символьно решающую задачу принятия решений методом log-чебышёвской аппроксимации в max-алгебре.
		Почему C++?
		Встраиваемость: например, библиотеку на \texttt{C++} можно использовать в языках \texttt{Python}, \texttt{R}, \texttt{Javascript} (через \texttt{NodeJS} или \texttt{wasm}) и многих других.
		Доступность библиотек с открытым исходным кодом позволяет использовать библиотеку на бóльшем числе процессорных архитектур и операционных систем.
		Данное решение более специализировано за счёт чего можно достичь бóльшей производительности и меньшего веса программы, по сравнению с написанием программы для существующих систем компьютерной алгебры.
		Результаты:
		Были релизованы max-+ и max-$\times$-алгебры и добавлена их поддержка в библиотеку линейной алгебры \texttt{Eigen} (версия 3.3.8).
		Все вычисление проводятся символьно (при помощи библиотеки \texttt{GiNaC} версии 1.8.0).
		Был реализован набор функций (напр. тропический определитель, спектральный радиус, оператор Клини), используемых в алгоритме решения задачи.
		Многокритериальная задача принятия решений может быть решена полностью при помощи программы.
	}
\end{frame}

\defverbatim[colored]\maxalglst{
\begin{lstlisting}[language=C++,basicstyle=\tiny\ttfamily,keywordstyle=\color{blue}]
Template<typename Op>
class MaxAlgebra {
	public:
	GiNaC::ex value;
	MaxAlgebra() : value(){};
	friend MaxAlgebra operator+(const MaxAlgebra& lhs, const MaxAlgebra& rhs);
	friend MaxAlgebra operator*(const MaxAlgebra& lhs, const MaxAlgebra& rhs);
	[...]
};
\end{lstlisting}
}

\defverbatim[colored]\maxplustimeslst{
\begin{lstlisting}[language=C++,basicstyle=\tiny\ttfamily,keywordstyle=\color{blue}]
	
using MaxTimes = MaxAlgebra<std::multiplies<void>>;
using MaxPlus = MaxAlgebra<std::plus<void>>;
[...]
\end{lstlisting}
}

\subsection{Внутреннее устройство}
\begin{frame}

	\begin{enumerate}
		\item Класс \texttt{MaxAlgebra} реализующий операции в max-алгебре.
		\maxalglst
		\item Две специализации класса реализующие операции в max-+ и max-$\times$-алгебре.
		\maxplustimeslst
		\item Расширение для библиотеки \texttt{Eigen}, позволяющее использовать \texttt{MaxAlgebra} в качестве элементов матрицы.
		\item Набор функций, участвующих в алгоритме решения задачи: нахождение тропического определителя, спектрального радиуса, оператор Клини, нахождение линейнонезависимых векторов и т. д.
	\end{enumerate}
	\note{
	Библиотека представляет собой:
    Класс \texttt{MaxAlgebra} реализующий операции в max-алгебре.
	Две специализации класса реализующие операции в max-+ и max-$\times$-алгебре.
	Расширение для библиотеки \texttt{Eigen}, позволяющее использовать \texttt{MaxAlgebra} в качестве элементов матрицы.
	Набор функций, участвующих в алгоритме решения задачи: нахождение тропического определителя, спектрального радиуса, оператор Клини, нахождение линейнонезависимых векторов и т. д.
    }
\end{frame}

\section{Заключение}
\begin{frame}
	\begin{itemize}
		\item В результате проделанной работы были выполнены поставленные цели.
		\item Была написана программа, решающая многокритериальную задачу принятия решений.
	\end{itemize}

	В дальнейшем работу планируется расширить:
	\begin{itemize}
	\item Протестировать скорость работы программы относительно существующих решений.
	\item Сделать более удобный интерфейс.
	\end{itemize}
	
	\note{
		В итоге была написана библиотека на C++ и программа решающая задачу многокритериального сравнения.
		В будущем, хотелось бы протестировать скорость работы программы относительно существующей реализации моего научного руководителя на Matlab (которое не является публичным),
		протестировать скорость работы программы относительно существующих решений.
		Также хотелось бы сделать более удобный интерфейс, чтобы им могли пользоваться практики, желающие пользоваться этим методом.
	}
\end{frame}

\begin{frame}{Список литературы}
	\begin{itemize}
		\item Saaty T. L., Vargas L. G. Comparison of eigenvalue, logarithmic least squares and least squares methods in estimating ratios // Mathematical Modelling. — 1984. — с. 309—324.
		\item Кривулин Н. К., Агеев В. А. Методы тропической оптимизации в многокритериальных задачах оценки альтернатив на основе парных сравнений // Вестник Санкт­ Петербургского университета. Прикладная математика. Информатика. Процессы управления. — 2019. — дек. — с. 472—488.
	\end{itemize}
\end{frame}

\end{document}
