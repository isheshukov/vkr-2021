\usepackage[utf8]{inputenc}
\usepackage[T2A]{fontenc}
\usepackage[english, russian]{babel}
\usepackage[a4paper,
  mag=1000,
  includefoot,
  left=3cm,
  right=1.5cm,
  top=2cm,
  bottom=2cm,
  headsep=1cm,
  footskip=1cm]{geometry}
\usepackage{longtable}
\usepackage{subcaption}
\usepackage{fancyvrb}
\usepackage{float}
\usepackage{listings}
\lstset{extendedchars=\true}
\usepackage{xcolor}
\usepackage{amsmath, amsfonts, amsthm}
\usepackage{graphics}

\usepackage{csquotes}
\usepackage{nicefrac}
\usepackage{minted}
\usepackage[
  backend=biber,
  bibstyle=gost-numeric,
  citestyle=gost-numeric,
  hyperref=true,
  natbib=true,
  url=true,
]{biblatex}
\usepackage{listings}
\usepackage{bm}

\usepackage{csquotes}

\usepackage{mathtools}
\usepackage{romannum}
\hypersetup{unicode=true}
\addbibresource{./ref.bib}
\let\vec=\mathbf
\setcounter{tocdepth}{2}
\author{Илья Шешуков}
\date{\number\year}
%\title{Отчёт по научно-исследовательской работе}
\title{Бакалаврская работа}
\hypersetup{
 pdfauthor={Илья Шешуков},
 pdftitle={Курсовая работа},
 pdfkeywords={},
 pdfsubject={},
 pdfcreator={Emacs 26.3},
 pdflang={Russian}}

\newtheorem{utv}{Утверждение}

% Имена без курсива
\renewcommand*{\mkgostheading}[1]{#1}
% Правка записей типа thesis, чтобы дважды не писался автор
\DeclareBibliographyDriver{thesis}{
  \usebibmacro{bibindex}%
  \usebibmacro{begentry}%
  \usebibmacro{heading}%
  \newunit
  \usebibmacro{author}%
  \setunit*{\labelnamepunct}%
  \usebibmacro{thesistitle}%
  \setunit{\respdelim}%
  %\printnames[last-first:full]{author}%Вот эту строчку нужно убрать, чтобы автор диссертации не дублировался
  \newunit\newblock
  \printlist[semicolondelim]{specdata}%
  \newunit
  \usebibmacro{institution+location+date}%
  \newunit\newblock
  \usebibmacro{chapter+pages}%
  \newunit
  \printfield{pagetotal}%
  \newunit\newblock
  \usebibmacro{doi+eprint+url+note}%
  \newunit\newblock
  \usebibmacro{addendum+pubstate}%
  \setunit{\bibpagerefpunct}\newblock
  \usebibmacro{pageref}%
  \newunit\newblock
  \usebibmacro{related:init}%
  \usebibmacro{related}%
  \usebibmacro{finentry}
}

\theoremstyle{definition}
\newtheorem{definition}{Определение}[section]

\DeclareMathOperator{\tr}{tr}
\DeclareMathOperator{\Tr}{Tr}


\institution{
    Санкт-Петербургский государственный университет \\
    Прикладная математика и информатика \\
    Вычислительная стохастика и статистические модели
}

\topic{\normalfont\scshape%
Разработка вычислительных алгоритмов и программных средств поддержки принятия решений
}

% Научный руководитель
\sa       {Н.\,К.~Кривулин}
\sastatus {доктор физ.-мат. наук, профессор}

% Город и год
\city{Санкт-Петербург}
