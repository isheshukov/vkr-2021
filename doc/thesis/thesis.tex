% !TeX spellcheck = ru_RU
\documentclass[specialist,
	substylefile = spbu_report.rtx,
	subf,href,colorlinks=true, 12pt]{disser}

\usepackage[utf8]{inputenc}
\usepackage[T2A]{fontenc}
\usepackage[english, russian]{babel}
\usepackage[a4paper,
  mag=1000,
  includefoot,
  left=3cm,
  right=1.5cm,
  top=2cm,
  bottom=2cm,
  headsep=1cm,
  footskip=1cm]{geometry}
\usepackage{longtable}
\usepackage{subcaption}
\usepackage{fancyvrb}
\usepackage{float}
\usepackage{listings}
\lstset{extendedchars=\true}
\usepackage{xcolor}
\usepackage{amsmath, amsfonts, amsthm}
\usepackage{graphics}
\usepackage{csquotes}
\usepackage{nicefrac}
\usepackage{minted}
\usepackage[
  backend=biber,
  bibstyle=gost-numeric,
  citestyle=gost-numeric,
  hyperref=true,
  natbib=true,
  url=true,
]{biblatex}

\usepackage{mathtools}
\usepackage{romannum}
\hypersetup{unicode=true}
\addbibresource{./ref.bib}
\let\vec=\mathbf
\setcounter{tocdepth}{2}
\author{Илья Шешуков}
\date{\number\year}
\title{Отчёт по научно-исследовательской работе}
\hypersetup{
 pdfauthor={Илья Шешуков},
 pdftitle={Курсовая работа},
 pdfkeywords={},
 pdfsubject={},
 pdfcreator={Emacs 26.3},
 pdflang={Russian}}

\newtheorem{utv}{Утверждение}

% Имена без курсива
\renewcommand*{\mkgostheading}[1]{#1}
% Правка записей типа thesis, чтобы дважды не писался автор
\DeclareBibliographyDriver{thesis}{
  \usebibmacro{bibindex}%
  \usebibmacro{begentry}%
  \usebibmacro{heading}%
  \newunit
  \usebibmacro{author}%
  \setunit*{\labelnamepunct}%
  \usebibmacro{thesistitle}%
  \setunit{\respdelim}%
  %\printnames[last-first:full]{author}%Вот эту строчку нужно убрать, чтобы автор диссертации не дублировался
  \newunit\newblock
  \printlist[semicolondelim]{specdata}%
  \newunit
  \usebibmacro{institution+location+date}%
  \newunit\newblock
  \usebibmacro{chapter+pages}%
  \newunit
  \printfield{pagetotal}%
  \newunit\newblock
  \usebibmacro{doi+eprint+url+note}%
  \newunit\newblock
  \usebibmacro{addendum+pubstate}%
  \setunit{\bibpagerefpunct}\newblock
  \usebibmacro{pageref}%
  \newunit\newblock
  \usebibmacro{related:init}%
  \usebibmacro{related}%
  \usebibmacro{finentry}
}

\theoremstyle{definition}
\newtheorem{definition}{Определение}[section]


\institution{
    Санкт-Петербургский государственный университет \\
    Прикладная математика и информатика \\
    Вычислительная стохастика и статистические модели
}

\topic{\normalfont\scshape%
Разработка вычислительных алгоритмов и программных средств поддержки принятия решений
}

% Научный руководитель
\sa       {Н.\,К.~Кривулин}
\sastatus {доктор физ.-мат. наук, профессор}

% Город и год
\city{Санкт-Петербург}


\begin{document}
\pagenumbering{arabic}
\maketitle
\newpage
\tableofcontents
\intro

Задача принятия решения на основе парных сравнений состоит в том, чтобы по результатам попарных сравнений $n$ альтернатив определить абсолютный приоритет каждой альтернативы. Подобные задачи возникают во множестве областей: социологии, маркетинге, менеджменте.

Одним из методов решения подобных задач является метод log-чебышёвской аппроксимации. Преимущество этого метода в том, что решения получают в аналитическом виде, а не в численном. Однако, пока не существует общедоступных программных средств, которые бы решали данную задачу на компьютере.

Поэтому в работе будет представлено программное средство, позволяющее решать задачи парных сравнений методом log-чебышёвской аппроксимации.

Работа была начата в этом году, в ходе работы были изучены материалы на тему тро­пической оптимизации и было начато написание библиотеки для решения задачи принятия решений. Был приведён план
и пример решения многокритериальной задачи с использованием разрабатываемого программного средства.

\chapter{Задача парных сравнений}
\begin{definition}{Матрицей парных сравнений}
	называется матрица $\boldsymbol{A} = (a_{ij})$, элемент $a_{ij}$ которой показывает, во сколько раз альтернатива $i$ превосходит альтернативу $j$ и для которой верны условия
	\[
		a_{ij} = \nicefrac{1}{a_{ji}}, \quad a_{ij} > 0.
	\]
\end{definition}
\begin{definition}{Согласованной матрицей}
	называется матрица $\boldsymbol{A} = (a_{ij})$ такая, что выполняется свойство транзитивности. Т.~е.
	\[
		a_{ij} = a_{ik}a_{kj}, \quad i,j,k = 1,\dots,n.
	\]
\end{definition}
\begin{utv}
	Необходимым и достаточным условием~\cite{saaty1984} того, что матрица $\boldsymbol{A}$
	согласованна является существование вектора $\boldsymbol{x}$, называемого \textit{вектором абсолютных приоритетов}, такого, что
	\[
		a_{ij} = x_{i}/x_{j} \quad \forall i, j.
	\]
\end{utv}
На практике, матрицы парных сравнений обычно несогласованны, и поэтому возникает задача приближения матрицы парных сравнений согласованной матрицей.
Одним из способов решения (т.~н. $\log$-чебышёвская аппроксимация~\cite{krivulin2019}) этой задачи является нахождение вектора $\boldsymbol{x}$, который бы минимизировал следующее выражение
\[
	l_{\infty}\left(\boldsymbol{A}, \boldsymbol{x} \boldsymbol{x}^{-}\right)=\max _{1 \leq i, j \leq n}\left|\log a_{i j}-\log \frac{x_{i}}{x_{j}}\right|, \quad \boldsymbol{x}^{-} = \begin{pmatrix}x_1^{-1}&x_2^{-1} & \dots & x_n^{-1}\end{pmatrix},
\]
что сводится к нахождению $\boldsymbol{x}$ такого, что
\[
	\min _{\boldsymbol{x}>0} \max _{1 \leq i, j \leq n} \frac{a_{i j} x_{j}}{x_{i}}.
\]
Данная задача, называемая однокритериальной, может быть решена с использованием идемпотентной алгебры и max-алгебры в частности.
На языке max-алгебры решение задачи будет выглядеть следующим образом
\[
	\min _{\boldsymbol{x}>0} \bigoplus_{1 \leq i, j \leq n} x_{i}^{-1} a_{i j} x_{j},
\]
что можно дополнительно переписать как
\[
	\min _{\boldsymbol{x}>0} x^{-} A x.
\]

\begin{definition}{Max-алгебра}
	есть алгебра над множеством $\mathbb{R}_{+}=\{x \in \mathbb{R} \mid x \geq 0\}$ с операциями ${(\oplus, \times)}$, где $\oplus$ --- это максимум, а $\times$ --- стандартное умножение.
\end{definition}

\section{Некоторые свойства max-алгебры}
\begin{enumerate}
	\item Сложение идемпотентно
	      \[
		      x \oplus x = x \quad \forall x \in \mathbb{R}_+.
	      \]
	\item Вычитание не определено
	\item Экстремальное свойство сложения
	      \[
		      x \leq x \oplus y,\quad y \leq x \oplus y \quad \forall x, y \in \mathbb{R}_+.
	      \]
	\item Изотонность сложения и умножения
	      \[
		      x \leq y \Rightarrow x \oplus z \leq y \oplus z,\quad xz \leq yz \quad \forall x,y,z \in \mathbb{R}_+.
	      \]
	\item Антитонность обращения
	      \[
		      x \leq y \Rightarrow x^{-1} \geq y^{-1} \quad\forall x,y \in \mathbb{R}_+ \setminus \{0\}.
	      \]
	\item Эквивалентность неравенств
	      \[
		      x \oplus y \leq z \Leftrightarrow x \leq z,\quad y \leq z \quad \forall x,y,z \in \mathbb{R}_+.
	      \]
\end{enumerate}
\section{Некоторые объекты, используемые в решении задачи}

\begin{definition}{Тропический определитель матрицы}
	\[
		\Tr \boldsymbol{A} \coloneqq \tr\boldsymbol{A} \oplus \dots \oplus \tr\boldsymbol{A}^n.
	\]
\end{definition}
\begin{definition}{Спектральный радиус}
	\[
		\lambda \coloneqq \tr\boldsymbol{A} \oplus \dots \oplus \tr^{1/n}\boldsymbol{A}^n.
	\]
\end{definition}
\begin{definition}{Оператор Клини\\}
	Если $\Tr \boldsymbol{A} \leq 1$, то оператор Клини определяется следующим образом
	\[
		\boldsymbol{A}^* \coloneqq \boldsymbol{I} \oplus \boldsymbol{A} \oplus \dots \oplus \boldsymbol{A}^{n-1}.
	\]
\end{definition}

\section{Многокритериальная задача парных сравнений}
Программное средство, разрабатываемое в рамках этой работы, поддерживает и многокритериальные задачи.

\subsection{Постановка многокритериальной задачи}

Пусть $n$ альтернатив сравниваются по $m$ критериям.
Матрица $\boldsymbol{A}_k$ --- матрица попарных сравнений по критерию с номером $k = 1, \dots, n$.
Матрица $\boldsymbol{C} = (c_{ij})$ --- матрица, показывающая, во сколько раз критерий с номером $i$ важнее критерия с номером $j$. Необходимо построить вектор абсолютных альтернатив по матрицам $\boldsymbol{A}_k$ и $\boldsymbol{C}$.
Определим функцию
$$
	f_{k}(\boldsymbol{x})=d\left(\boldsymbol{A}_{k}, \boldsymbol{x} \boldsymbol{x}^{-}\right)
$$ и тогда задача сводится к нахождению
$$
	\min _{\boldsymbol{x}>\mathbf{0}}\left(f_{1}(\boldsymbol{x}), \ldots, f_{m}(\boldsymbol{x})\right).
$$

В терминах тропической математики задача принимает вид
$$
	\min _{\boldsymbol{x}>\mathbf{0}} \boldsymbol{x}^{-} \boldsymbol{A} \boldsymbol{x}, \quad \text { где } \quad \boldsymbol{A}=\bigoplus_{1 \leq k \leq m} w_{k} \boldsymbol{A}_{k},
$$
где вектор весов $\boldsymbol{w}=\left(w_{k}\right)$ находится как решение задачи
\[
	\min _{\boldsymbol{w}>0} \boldsymbol{w}^{-} \boldsymbol{C} \boldsymbol{w}.
\]
Все решения задачи определения весов записываются в виде
\[
	\boldsymbol{w} = (\lambda^{-1}\boldsymbol{C})^*\boldsymbol{v}, ~\boldsymbol{v} > 0,
\]
где $\lambda$ --- спектральный радиус матрицы $\boldsymbol{C}$. Аналогично, если $\mu$ --- спектральный радиус матрицы $\boldsymbol{A}$, то все решения задачи оценки рейтингов альтернатив имеют вид:
\[
	\boldsymbol{x} = (\mu^{-1}\boldsymbol{A})^*\boldsymbol{u},~\boldsymbol{u} > 0.
\]


\subsection{План решения многокритериальной задачи}
\begin{enumerate}
	\item По матрице $\boldsymbol{C}$ определяем вектор весов критериев:
	      \[
		      \boldsymbol{w} = (\mu^{-}\boldsymbol{C})^*\boldsymbol{u},\quad  \boldsymbol{u} > 0,\quad \mu = \tr(\boldsymbol{C_1}) \oplus\ldots\oplus \tr^{1/m}(\boldsymbol{C_m}).
	      \]
	\item Если полученный вектор $\boldsymbol{u}$ не единственный, то находятся наихудший и наилучший дифференцирующий вектор весов:
	      \[
		      \boldsymbol{w}_1 = (\delta^-\boldsymbol{11}^\mathrm{T} \oplus \mu^{-1}\boldsymbol{C})^*\boldsymbol{v_1},\quad  \boldsymbol{v_1} > 0,\quad \delta = \boldsymbol{1}^\mathrm{T}(\mu^{-}C)^*\boldsymbol{1},
	      \]
	      и
	      \[
		      \boldsymbol{w}_2 =\boldsymbol{P}(\boldsymbol{I}\oplus\boldsymbol{P}^{-}_{lk}\boldsymbol{P})\boldsymbol{v_2},\quad \boldsymbol{v_2} > 0,
	      \]
	      для которого матрица $\boldsymbol{P}$ получена из матрицы $(\mu^{-1}\boldsymbol{C})^*$ вычеркиванием линейно-зависимых столбцов, матрица $\boldsymbol{P}_{lk}$ --- из матрицы $\boldsymbol{P} = (p_{ij})$ заменой на ноль всех элементов, кроме $p_{lk}$, а индексы $k$ и $l$ определяются, исходя из условий
	      \[
		      k = \underset{j}{\arg\max}~\boldsymbol{1}^\mathrm{T}\boldsymbol{p}_j\boldsymbol{p}^-_j\boldsymbol{1},\quad l = \underset{i}{\arg\max} ~p^{-1}_{ik}.
	      \]
	\item С помощью векторов $\boldsymbol{w}_1 = (w^{(1)}_i)$ и $\boldsymbol{w}_2 = (w^{(2)}_i)$ составляются взвешенные
	      суммы матриц парных сравнений:
	      \begin{align*}
		      \boldsymbol{D}_1 & = w^{(1)}_1\boldsymbol{A_1}\oplus\ldots\oplus w^{(1)}_m\boldsymbol{A_m}, \\
		      \boldsymbol{D}_2 & =w^{(2)}_1\boldsymbol{A_1}\oplus\ldots\oplus w^{(2)}_m\boldsymbol{A_m}.
	      \end{align*}
	\item Вычисляется вектор рейтингов альтернатив для матрицы $\boldsymbol{D_1}$
	      \[
		      \boldsymbol{x} = (v^{-1}_1\boldsymbol{D_1})^*\boldsymbol{u}_1,\quad \boldsymbol{u}_1 > 0,\quad v_1 = \tr\boldsymbol{D}_1\oplus\dots\oplus \tr^{1/n}\boldsymbol{D}_1^n.
	      \]
	\item Если полученный вектор не единственный, то вместо него ищется наихудший дифференцирующий вектор
	      \[
		      \boldsymbol{x}_1 = (\delta^{-1}_1\boldsymbol{11}^\mathrm{T}\oplus v^{-1}_1\boldsymbol{D}_1)^*\boldsymbol{u_1},\quad \boldsymbol{u}_1 > 0,\quad \delta_1 = \boldsymbol{1}^\mathrm{T}(v^{-1}\boldsymbol{D_1})^*\boldsymbol{1}.
	      \]
	\item Вычисляется вектор рейтингов альтернатив для матрицы $\boldsymbol{D_2}$
	      \[
		      \boldsymbol{x}_2 = (v^{-1}_2\boldsymbol{D}_2)^*\boldsymbol{u}_2,\quad \boldsymbol{u_2} > 0,\quad v_2 = \tr\boldsymbol{D}_2\oplus\ldots\oplus\tr^{1/n}(\boldsymbol{D}^n_2).
	      \]

	\item Если этот вектор не единственный, то вместо него рассматриваем наилучший дифференцирующий вектор
	      \[
		      \boldsymbol{x}_2 =\boldsymbol{S}(\boldsymbol{I}\oplus\boldsymbol{S}^-_{lk}\boldsymbol{S})\boldsymbol{u}_2,~ \boldsymbol{u}_2 > 0,
	      \]

	      где матрица $\boldsymbol{S}$ получена из матрицы $(v^{-1}_2\boldsymbol{D}_2)^*$ вычеркиванием линейно-зависимых столбцов, матрица $\boldsymbol{S}^-_{lk}$ --- из матрицы $S = (s_{ij})$ обращением
	      в нуль всех элементов, кроме $s_{lk}$, а индексы $k$ и $l$ определяются, исходя из условий
	      \[
		      k = \underset{j}{\arg\max}~\boldsymbol{1}^\mathrm{T}\boldsymbol{s}_j\boldsymbol{s}^-_j\boldsymbol{1},~l = \underset{i}{\arg\max} ~s^-_{ik}.
	      \]
\end{enumerate}


\section{Цель работы}
Требуется написать библиотеку и программу на языке \texttt{C++}, символьно решающую одно- и многокритериальную задачу принятия решений методом log-чебышёвской аппроксимации в max-алгебре.

\chapter{Ход работы}
\section{Выполнено}
\begin{enumerate}
	\item Было написано расширение для библиотеки линейной алгебры \texttt{Eigen} \cite{eigenweb} (версия 3.3.8), позволяющее работать с символьными вычислениями в max-алгебре.
	\item Был реализован ряд функций, используемых в алгоритме решения задачи. Таких как
        нахождение тропического определителя, спектрального радиуса, оператор Клини, нахождение линейнонезависимых векторов и других.
	\item Задача многокритериального сравнения может быть решена при помощи библиотеки.
\end{enumerate}

\section{Пример решения многокритериальной задачи}
\subsection{Условие}

\[
	\boldsymbol{C} = \begin{pmatrix}
		1 & 1/3 \\
		3 & 1
	\end{pmatrix}, \quad
	\boldsymbol{A}_1 = \begin{pmatrix}
		1   & 3 & 1/3 \\
		1/3 & 1 & 1   \\
		3   & 1 & 1
	\end{pmatrix}, \quad
	\boldsymbol{A}_2 =  \begin{pmatrix}
		1   & 1/3 & 5 \\
		3   & 1   & 7 \\
		1/5 & 1/7 & 1
	\end{pmatrix}.
\]
\subsection{Решение}
Найдём спектральный радиус матрицы \(\boldsymbol{C}\):

\[
	\lambda=\bigoplus_{k=1}^{2} \tr^{1 / k}\left(\boldsymbol{C}^{k}\right) = \tr\boldsymbol{C} \oplus \sqrt{\tr\boldsymbol{C}^2} =
	\tr\begin{pmatrix}1&1/3\\3&1\end{pmatrix} \oplus \sqrt{\tr\begin{pmatrix}1&1/3\\3&1\end{pmatrix}} = 1.
\]
В нашем случае, матрица Клини равна самой матрице \(\boldsymbol{C}\).
\[
	\boldsymbol{C}^* = I \oplus \boldsymbol{C} = \boldsymbol{C}.
\]
Тогда
\[
	(\lambda^{-1}\boldsymbol{C})^*\boldsymbol{v} = (\lambda^{-1})\boldsymbol{C}\boldsymbol{v} = \boldsymbol{C}\boldsymbol{v}, \quad \boldsymbol{v} > 0.
\]
Столбцы матрицы коллинеарны, поэтому просто возьмём первый из них. Вектор весов критериев
\[
	\boldsymbol{w} = \begin{pmatrix}1 & 3\end{pmatrix}.
\]
Вычислим матрицу \(\boldsymbol{B}\):
\[
	\boldsymbol{B}=\bigoplus_{k=1}^{2} w_{k} \boldsymbol{A}_{k} = \begin{pmatrix}1&3&1/3\\1/3&1&1\\3&1&1\end{pmatrix} \oplus 3\begin{pmatrix}1&1/3&5\\3&1&7\\1/5&1/7&1\end{pmatrix} = \begin{pmatrix}3&3&15\\ 9&3&21\\ 3&1&3\\\end{pmatrix}.
\]
Спектральный радиус этой матрицы равен \(\mu = \sqrt{45}\).
Тогда вектор рейтингов альтернатив:
\begin{align*}
	\boldsymbol{x} = \boldsymbol{S}\boldsymbol{u} = \left(\frac{1}{\sqrt{45}}B\right)^*\boldsymbol{u} & =
	\boldsymbol{I} \oplus \frac{1}{\sqrt{45}} \boldsymbol{B} \oplus (\frac{1}{\sqrt{45}} \boldsymbol{B})^2 =
	\begin{pmatrix}1& 0& 0\\0& 1& 0\\0& 0& 1\end{pmatrix}
	\oplus
	\begin{pmatrix}
		1/\sqrt{5} & 1/\sqrt{5}     & \sqrt{5}   \\
		3/\sqrt{5} & 1/\sqrt{5}     & 7/\sqrt{5} \\
		1/\sqrt{5} & 1/(3 \sqrt{5}) & 1/\sqrt{5} \\
	\end{pmatrix}                                          \\
	                                                           & \oplus
	\begin{pmatrix}1& 1/3& 7/5\\7/5& 3/5& 3\\1/5& 1/5& 1\end{pmatrix} =
	\begin{pmatrix} 1&1/\sqrt{5}&\sqrt{5}\\ 7/5&1&7/\sqrt{5}\\ 1/\sqrt{5}&1/5&1 \end{pmatrix}\boldsymbol{u}.
\end{align*}

Здесь только первый и второй столбцы являются линейно независимыми, значит будем искать наихудший и наилучший дифференцирующий вектор.

\subsubsection{Наихудший дифференцирующий вектор}
\[
	\delta = 1^\mathrm{T}(\mu^{-1}\boldsymbol{B})^*1 = 7/\sqrt{5}.
\]

\begin{align*}
	\boldsymbol{x_2} & = (\delta^{-1}\boldsymbol{1}\boldsymbol{1}^\mathrm{T} \oplus \mu^{-1}\boldsymbol{B})^*\boldsymbol{u} = \begin{pmatrix}\sqrt{45}/15 & \sqrt{45}/15 & \sqrt{45}/3\\ \sqrt{45}/5 & \sqrt{45}/15 & 7\sqrt{45}/15 \\ \sqrt{45}/15 & \sqrt{45}/45 & \sqrt{45}/15 \end{pmatrix}^*u =                   \\
	                 & = \boldsymbol{I} \oplus (\delta^{-1}\boldsymbol{1}\boldsymbol{1}^\mathrm{T} \oplus \mu^{-1}\boldsymbol{B}) \oplus (\delta^{-1}\boldsymbol{1}\boldsymbol{1}^\mathrm{T} \oplus \mu^{-1}\boldsymbol{B})^2 = \\
	                 & = \begin{pmatrix} 1&0&0 \\ 0&1&0 \\ 0&0&1 \end{pmatrix}
	\oplus
	\begin{pmatrix}1/\sqrt{5}&1/\sqrt{5}&\sqrt{5}\\3/\sqrt{5}&1/\sqrt{5}&7/\sqrt{5}\\1/\sqrt{5}&3\sqrt{5}/7&1/\sqrt{5}\end{pmatrix}
	\oplus
	\begin{pmatrix}1&5/7&7/5\\7/5&1&3\\3/7&1/5&1\end{pmatrix}                                                                                     \\
	                 & = \begin{pmatrix} 1&5/7&\sqrt{5}\\ 7/5&1&7/\sqrt{5}\\ 1/\sqrt{5}&\sqrt{5}/7&1 \end{pmatrix}\boldsymbol{u}, \quad \boldsymbol{u} > 0.
\end{align*}

Тут все столбцы коллинеарны. Возьмём первый и нормируем:

\begin{align*}
	\begin{pmatrix}0.3512\\0.4917\\0.1570\end{pmatrix}.
\end{align*}

Порядок альтернатив: (\Romannum{2}) \(\succ\) (\Romannum{1}) \(\succ\) (\Romannum{3}).
\subsubsection{Наилучший дифференцирующий вектор}
\label{sec:org896b7a9}
Найденный нами наихудший дифференцирующий вектор, это первый столбец матрицы $\boldsymbol{S}$. Можем предположить, что наилучший дифференцирующий вектор окажется вторым столбцом этой матрицы. Давайте это проверим.

Из предыдущего получили, что \(\boldsymbol{S} = \begin{pmatrix}1&1/\sqrt{5}\\7/5&1\\1/\sqrt{5}&1/5\end{pmatrix}\).
Для нахождения наилучшего дифференцирующего вектора найдём число \(\Delta\):

\[
	\Delta = \mathbf{1}^\mathrm{T} \boldsymbol{S} \boldsymbol{S}^{-} \mathbf{1} = \begin{pmatrix}7/5&1\end{pmatrix}\begin{pmatrix}\sqrt{5}\\5\end{pmatrix} = 5.
\]
Все наилучшие дифференцирующие решения имеют вид

\[
	\boldsymbol{x}_{1}=\boldsymbol{S}\left(\boldsymbol{I} \oplus \boldsymbol{S}_{l k}^{-} \boldsymbol{S}\right) \boldsymbol{v}, \quad \boldsymbol{v}>\mathbf{0},
\]
где индексы \(k\) и \(l\) определяются из условия

\[
	\mathbf{1}^\mathrm{T} \boldsymbol{s}_{k} s_{l k}^{-1}=\Delta = 5.
\]
Это условие выполняется только при \(k=2\) и \(l=3\). Тогда наилучшие решения генерируют столбцы матрицы:

\begin{align*}
	\boldsymbol{x}_{1} & =\boldsymbol{S}\left(\boldsymbol{I} \oplus \boldsymbol{S}_{3,2}^{-} \boldsymbol{S}\right) \boldsymbol{v} = \begin{pmatrix}1&1/\sqrt{5}\\7/5&1\\1/\sqrt{5}&1/5\end{pmatrix} \left( \begin{pmatrix}1&0\\0&1\end{pmatrix} \oplus \begin{pmatrix}0&0&0\\0&0&5\end{pmatrix}\begin{pmatrix}1&  1/\sqrt{5}\\7/5&1\\1/\sqrt{5}&1/5\end{pmatrix} \right)\boldsymbol{v} \\
	                   & = \begin{pmatrix}1&1/\sqrt{5}\\\sqrt{5}&1\\1/\sqrt{5}&1/5\end{pmatrix}\boldsymbol{v}.
\end{align*}
Столбцы этой матрицы коллинеарны, возьмём второй.

Как и предполагалось, наилучший дифференцирующий вектор коллинеарен второму столбцу матрицы \(\boldsymbol{S}\).

\[
	x_1 = \begin{pmatrix}
		1/\sqrt{5} \\1\\1/5
	\end{pmatrix} \approx \begin{pmatrix}
		0.4472 \\ 1 \\ 0.2
	\end{pmatrix}.
\]

Порядок альтернатив: (\Romannum{2}) \(\succ\) (\Romannum{1}) \(\succ\) (\Romannum{3}).

\textbf{Общий ответ}: (\Romannum{2}) \(\succ\) (\Romannum{1}) \(\succ\) (\Romannum{3}).

\subsection{Использование библиотеки для вычислений}
Приведём пример использования библиотеки для получения вышеприведённых вычислений на компьютере (без нахождения дифференцирующих векторов, другой пример с нахождением дифференцирующих векторов может быть найден в Приложении \ref{appendix:full}).

Задание условий:
\begin{verbatim}
MatrixDn C(2, 2);
C << 1, ma(1,3), 
     3, 1;
		
MatrixDn A1(3, 3);
A1 <<  1, 3, ma(1,3), 
ma(1, 3), 1, 1, 
       3, 1, 1;
MatrixDn A2(3, 3);
A2 <<  1,      1/3, 5, 
       3,        1, 7, 
ma(1, 5), ma(1, 7), 1;
\end{verbatim}

Вычисление спектрального радиуса и матрицы Клини:
\begin{verbatim}
auto lambda = spectral_radius(C);

MatrixDn C_ = 1 / lambda * C;
auto w = cleany(C_);
\end{verbatim}
Вывод:
\begin{verbatim}
Вычисление спектрального радиуса
lambda = 1

Вычисление (\lambda^{-1}C)^*
w = [1, 1/3]
	[3, 1]
\end{verbatim}

Вычисление матрицы $\boldsymbol{B}$ и её спектрального радиуса:
\begin{verbatim}
auto B = (A1 + 3 * A2).eval();
auto mu = spectral_radius(B);
\end{verbatim}
Вывод:
\begin{verbatim}
B = [3, 3, 15]
	[9, 3, 21]
	[3, 1, 3]

Вычисление спектрального радиуса B
mu = sqrt(45)
\end{verbatim}

Вычисление матрицы $\boldsymbol{S}$ и формулы для получения вектора рейтингов альтернатив:
\begin{verbatim}
auto S_ = (1/mu*B).eval();
auto S = cleany(S_);
\end{verbatim}
Вывод:
\begin{verbatim}
S = [1, 1/15*sqrt(45), 1/3*sqrt(45)]
	[7/5, 1, 7/15*sqrt(45)]
	[1/15*sqrt(45), 1/5, 1]

\end{verbatim}

Вычисления совпадают с теми, что были получены ручными вычислениями.

\subsection{Внутреннее устройство библиотеки}
Библиотека представляет собой
\begin{enumerate}
	\item Класс \texttt{MaxAlgebra<T>} реализующий операции в max-алгебре.

\begin{lstlisting}[language=C++,basicstyle=\footnotesize\ttfamily,keywordstyle=\color{blue}]
Template<typename Op>
class MaxAlgebra {
 public:
  GiNaC::ex value;
  MaxAlgebra() : value(){};
  ...
  friend MaxAlgebra operator+(const MaxAlgebra& lhs, const MaxAlgebra& rhs);
  friend MaxAlgebra operator*(const MaxAlgebra& lhs, const MaxAlgebra& rhs);
  friend MaxAlgebra operator/(const MaxAlgebra& lhs, const MaxAlgebra& rhs);
  friend bool operator<(const MaxAlgebra& lhs, const MaxAlgebra& rhs);
  friend bool operator==(const MaxAlgebra& lhs, const MaxAlgebra& rhs);
  friend bool operator>(const MaxAlgebra& lhs, const MaxAlgebra& rhs);
  friend MaxAlgebra pow(const MaxAlgebra& lhs, const MaxAlgebra& rhs);
  friend std::ostream& operator<<(std::ostream& out, const MaxAlgebra& val);
  ...
  MaxAlgebra& operator=(const MaxAlgebra& rhs);
  MaxAlgebra& operator+=(const MaxAlgebra& rhs);
  MaxAlgebra& operator*=(const MaxAlgebra& rhs);
  MaxAlgebra& operator/=(const MaxAlgebra& rhs);
  MaxAlgebra& abs(const MaxAlgebra& rhs);
};
\end{lstlisting}
\end{enumerate}

\begin{enumerate}
 \setcounter{enumi}{1}
	\item Две специализации класса реализующие операции в max-+ и max-$\times$-алгебре.
\begin{lstlisting}[language=C++,basicstyle=\footnitesize\ttfamily,keywordstyle=\color{blue}]

using MaxTimes = MaxAlgebra<std::multiplies<void>>;
using MaxPlus = MaxAlgebra<std::plus<void>>;

MaxPlus operator*(const MaxPlus& lhs, const MaxPlus& rhs);
MaxPlus operator/(const MaxPlus& lhs, const MaxPlus& rhs);
...
MaxTimes operator*(const MaxTimes& lhs, const MaxTimes& rhs);
MaxTimes operator/(const MaxTimes& lhs, const MaxTimes& rhs);
\end{lstlisting}
\end{enumerate}

\begin{enumerate}
  \setcounter{enumi}{2}
		\item Расширение для библиотеки \texttt{Eigen}, позволяющее использовать \texttt{MaxAlgebra<T>} в качестве элементов матрицы.
		\item Набор функций, участвующих в алгоритме решения задачи: нахождение тропического определителя, спектрального радиуса, оператор Клини, нахождение линейнонезависимых векторов и т. д.
\end{enumerate}
\subsection{Список реализованных функций}

\begin{enumerate}
	\item Арифметические операции над числами в max-+ и max-$\times$-алгебре: $\oplus$, $\times$, возведение в степень и т. д.
	\item Операции над векторами и матрицами: перемножение, транспонирование, возведение в степень и т. д.
	\item Функции, необходимые в решении задачи: спектральный радиус, получение матрицы Клини, нахождение коэффициентов в задаче нахождение наилучшего дифференцирующего вектора.
\end{enumerate}


\conclusion
\section{Планируется сделать}
\begin{enumerate}
	\item Сделать более удобный интерфейс.
	\item Протестировать скорость работы программы относительно существующих решений.
\end{enumerate}

\addcontentsline{toc}{chapter}{Список литературы}
\printbibliography

\appendix
\cleardoublepage\makeatletter\@openrightfalse\makeatother
\chapter{Примеры работы программы}
\section{Пример 1}
Объявление двух матриц $A_{1}$ и $A_{2}$ размера $3\times 3$:
\begin{lstlisting}[language=C++,basicstyle=\ttfamily,keywordstyle=\color{red}]
 MatrixDn A1(3, 3);
 MatrixDn A2(3, 3);

 A1 << ex(sin(1)), 3, ex(1) / 3,
        ex(1) / 3, 1, 1,
                3, 1, 1;
 A2 <<   0, ex(1) / 3, 5,
         3,         1, 7,
 ex(1) / 5, ex(1) / 7, 1;
  \end{lstlisting}
Сумма и произведение этих матриц:
\begin{lstlisting}[language=C++,basicstyle=\ttfamily,keywordstyle=\color{red}]
 std::cout << "A1+A2=\n" << A1+A2 << std::endl;
 std::cout << "A1*A2=\n" << A1*A2 << std::endl;
  \end{lstlisting}
Вывод:
\begin{lstlisting}[language=bash,basicstyle=\ttfamily,keywordstyle=\color{red}]
 A1 + A2 =               A1 * A2 =
   \sin(1) 3 5             9  3 21
   3       1 7             3  1  7
   3       1 1             3  1 15
  \end{lstlisting}
\section{Пример 2}
Объявление матрицы $C$ размера $4\times 4$:
\begin{lstlisting}[language=bash,basicstyle=\ttfamily,keywordstyle=\color{red}]
   MatrixDn C(4, 4);
   C <<  8, 3, 10, 6,
         7, 2,  1, 4,
        10, 8,  2, 2,
         4, 7,  6, 1;
  \end{lstlisting}
Вычисление спектрального радиуса:
\begin{lstlisting}[language=bash,basicstyle=\ttfamily,keywordstyle=\color{red}]
   std::cout
         << "Spectral radius of C = "
         << spectral_radius(C)
         << std::endl;
   \end{lstlisting}
Вывод:
\begin{lstlisting}[language=bash,basicstyle=\ttfamily,keywordstyle=\color{red}]
m^1=                      m^2=
 8  3 10  6               100  80  80  48
 7  2  1  4                56  28  70  42
10  8  2  2                80  30 100  60
 4  7  6  1                60  48  40  28
                          tr(m^2)^(1/2) = 10

m^3=                      m^4=
 800  640 1000  600       10000  8000  8000  4800
 700  560  560  336        5600  4480  7000  4200
1000  800  800  480        8000  6400 10000  6000
 480  320  600  360        6000  4800  4800  2880
tr(m^3)^(1/3) = 800^(1/3)   tr(m^4)^(1/4) = 10

Spectral radius of C = 10
\end{lstlisting}
\section{Пример 3. Решение многокритериальной задачи парного сравнения} \label{appendix:full}

\subsubsection{Условие}
$$
	\begin{array}{l}
		\boldsymbol{C}=\left(\begin{array}{cccccc}
				1     & 1 / 4 & 1 / 5 & 1 / 4 & 6 & 1 / 6 \\
				4     & 1     & 1 / 3 & 3     & 6 & 1 / 2 \\
				5     & 3     & 1     & 4     & 7 & 3     \\
				4     & 1 / 3 & 1 / 4 & 1     & 5 & 1 / 5 \\
				1 / 6 & 1 / 6 & 1 / 7 & 1 / 5 & 1 & 1 / 7 \\
				6     & 2     & 1 / 3 & 5     & 7 & 1
			\end{array}\right)                                                                                                                \\
		\boldsymbol{A}_{1}=\left(\begin{array}{ccc}
				1     & 5     & 8 \\
				1 / 5 & 1     & 5 \\
				1 / 8 & 1 / 5 & 1
			\end{array}\right), \quad \boldsymbol{A}_{2}=\left(\begin{array}{ccc}
				1     & 7     & 9 \\
				1 / 7 & 1     & 7 \\
				1 / 9 & 1 / 7 & 1
			\end{array}\right), \quad \boldsymbol{A}_{3}=\left(\begin{array}{ccc}
				1 & 1 / 7 & 1 / 9 \\
				7 & 1     & 1 / 7 \\
				9 & 7     & 1
			\end{array}\right), \\
		\boldsymbol{A}_{4}=\left(\begin{array}{ccc}
				1     & 3     & 5 \\
				1 / 3 & 1     & 4 \\
				1 / 5 & 1 / 4 & 1
			\end{array}\right), \quad \boldsymbol{A}_{5}=\left(\begin{array}{ccc}
				1     & 3     & 5 \\
				1 / 3 & 1     & 4 \\
				1 / 5 & 1 / 4 & 1
			\end{array}\right), \quad \boldsymbol{A}_{6}=\left(\begin{array}{ccc}
				1     & 7     & 9 \\
				1 / 7 & 1     & 7 \\
				1 / 9 & 1 / 7 & 1
			\end{array}\right)
	\end{array}
$$

\subsubsection{Код программы}

Задание матриц:
\begin{lstlisting}[language=c++,basicstyle=\small\ttfamily,keywordstyle=\color{red}]
MatrixDn C(6, 6);
C << 1, MaxTimes(1, 4), MaxTimes(1, 5), MaxTimes(1, 4), 6,
    MaxTimes(1, 6), 4, 1, MaxTimes(1, 3), 3, 6, MaxTimes(1, 2), 5, 3, 1,
    4, 7, 3, 4, MaxTimes(1, 3), MaxTimes(1, 4), 1, 5, MaxTimes(1, 5),
    MaxTimes(1, 6), MaxTimes(1, 6), MaxTimes(1, 7), MaxTimes(1, 5), 1,
    MaxTimes(1, 7), 6, 2, MaxTimes(1, 3), 5, 7, 1;

std::vector<MatrixDn>As;
MatrixDn A1(3, 3);
A1 << 1, 5, 8, ma(1, 5), 1, 5, ma(1, 8), ma(1, 5), 1;
MatrixDn A2(3, 3);
A2 << 1, 7, 9, ma(1, 7), 1, 7, ma(1, 9), ma(1, 7), 1;
MatrixDn A3(3, 3);
A3 << 1, ma(1, 7), ma(1, 9), 7, 1, ma(1, 7), 9, 7, 1;
MatrixDn A4(3, 3);
A4 << 1, 3, 5, ma(1, 3), 1, 4, ma(1, 5), ma(1, 4), 1;
MatrixDn A5(3, 3);
A5 << 1, 3, 5, ma(1, 3), 1, 4, ma(1, 5), ma(1, 4), 1;
MatrixDn A6(3, 3);
A6 << 1, 7, 9, ma(1, 7), 1, 7, ma(1, 9), ma(1, 7), 1;

As.push_back(A1);
As.push_back(A2);
As.push_back(A3);
As.push_back(A4);
As.push_back(A5);
As.push_back(A6);

const auto num_crit_mat = C.rows();
const auto num_comp_mat = A1.rows();
\end{lstlisting}

Вычисление спектрального радиуса:
\begin{lstlisting}[language=c++,basicstyle=\footnotesize\ttfamily]
auto lambda = spectral_radius(C);
\end{lstlisting}

Вычисление $(\lambda^{-1}C)^*$:
\begin{lstlisting}[language=c++,basicstyle=\footnotesize\ttfamily]
MatrixDn Calmoststar = (1 / lambda) * C;
auto w = cleany(Calmoststar);
\end{lstlisting}


Вычисление $\delta$:
\begin{lstlisting}[language=c++,basicstyle=\footnotesize\ttfamily]
auto delta = (MatrixDn::Ones(1, num_crit_mat) * w * MatrixDn::Ones(num_crit_mat, 1)).value();
\end{lstlisting}

Вычисление наихудшего дифференцирующего вектора:
\begin{lstlisting}[language=c++,basicstyle=\footnotesize\ttfamily]
MatrixDn W1m = (1 / delta * MatrixDn::Ones(num_crit_mat, num_crit_mat)) + Calmoststar;
auto w1 = cleany(W1m);
auto liw1 = linearly_independent_system(w1);
\end{lstlisting}

Получилась матрица из двух линейно независимых векторов. Попробуем получить \enquote{самый наихудший} вектор. Возьмём линейнонезависимые столбцы матрицы, нормируем по максимуму и сложим:
\begin{lstlisting}[language=c++,basicstyle=\footnotesize\ttfamily]
MatrixDn worst_vector = MatrixDn::Zero(liw1.rows(), 1);
for (int i = 0; i < liw1.cols(); ++i){
    worst_vector += liw1.col(i) / liw1.col(i).maxCoeff();
}
\end{lstlisting}


Получим взвешенную сумму:
\begin{lstlisting}[language=c++,basicstyle=\footnotesize\ttfamily]
MatrixDn D1 = MatrixDn::Zero(num_comp_mat, num_comp_mat);
for (int i = 0; i < worst_vector.rows(); ++i){
    D1 += worst_vector(i) * As[i];
}

auto nu_1 = spectral_radius(D1);
MatrixDn almost_x1 = (1 / nu_1) * D1;
auto x1 = cleany(almost_x1);
auto LI_x1 = linearly_independent_system(x1);
\end{lstlisting}

Вычисляем вектор рейтингов альтернатив матрицы $\mathbf{D}_{!}$.
Если матрица в переменной \textbf{LI_x1} состоит из одного столбца, то этот столбец является решением. В противном случае неединственности решения:

\begin{lstlisting}[language=c++,basicstyle=\footnotesize\ttfamily]
auto delta_1 = (MatrixDn::Ones(1, x1.rows()) * x1 * MatrixDn::Ones(x1.cols(), 1)).value();
MatrixDn almost_x1new = ((1/delta_1) * MatrixDn::Ones(x1.cols(), x1.rows())) + almost_x1;
auto x1new = cleany(almost_x1new);

auto LI_x1new = linearly_independent_system(x1new);
\end{lstlisting}

Вычисление наилучшего дифференцирующего вектора.
Матрица $P$ --- линейно-независимые столбцы матрицы $w$.
\begin{lstlisting}[language=c++,basicstyle=\footnotesize\ttfamily]
MatrixDn P = linearly_independent_system(w);
\end{lstlisting}

Получение коэффициентов $k$ и $l$:
\begin{lstlisting}[language=c++,basicstyle=\footnotesize\ttfamily]
auto bdfc = best_diff_vector_coefficients(P);
\end{lstlisting}


Вычисление $w_2$. На этом этапе вычислим $w_{2}$ для каждой пары коэфициентов из предыдущего шага, объединим в одну матрицу и найдём линейнонезависимую часть.
\begin{lstlisting}[language=c++,basicstyle=\footnotesize\ttfamily]
for (auto& [k, l]: bdfc){

    MatrixDn Plk_inverse = MatrixDn::Zero(P.rows(), P.cols());
    Plk_inverse(l, k) = 1 / P(l, k);
    auto t = P * (MatrixDn::Identity(P.cols(), P.cols()) +
                   Plk_inverse.transpose() * P);
    W2.conservativeResize(t.rows(), W2.cols() + t.cols());
    W2.rightCols(t.cols()) = t;
}
auto LI_W2 = linearly_independent_system(W2);
\end{lstlisting}

Вычислияем вектор рейтингов альтернатив для $\mathbf{D}_2$.
\begin{lstlisting}[language=c++,basicstyle=\footnotesize\ttfamily]
auto nu_2 = spectral_radius(D2);
MatrixDn almost_x2 = (1 / nu_2) * D2;

auto x2 = cleany(almost_x2);
auto LI_x2 = linearly_independent_system(x2);

if (LI_x2.cols() == 1){
  // Случай единственности
    std::cout << "x2 = " << LI_x2.format(MatlabFmt) << std::endl;
} else {
  // Случай неединственности
    auto S = LI_x2;
    auto bdfc = best_diff_vector_coefficients(S);
    MatrixDn WS;
    for (auto& [k, l]: bdfc){
        MatrixDn Slk_inverse = MatrixDn::Zero(S.rows(), S.cols());
        Slk_inverse(l, k) = 1 / S(l, k);
        auto t = S * (MatrixDn::Identity(S.cols(), S.cols()) +
                       Slk_inverse.transpose() * S);
        WS.conservativeResize(t.rows(), WS.cols() + t.cols());
        WS.rightCols(t.cols()) = t;
    }
    auto LI_WS = linearly_independent_system(WS);
  }
\end{lstlisting}

\subsubsection{Вывод программы}

Вычисление  спектрального радиуса:
\begin{lstlisting}[language=c++,basicstyle=\footnotesize\ttfamily]
	lambda = (360/7)^(1/5)
\end{lstlisting}

Вычисление $(\lambda^{-1}C)^*$:
\begin{lstlisting}[language=c++,basicstyle=\scriptsize\ttfamily]
	w = 
	[1, 1/20*(360/7)^(2/5), 1/60*(360/7)^(3/5), 1/4*(360/7)^(1/5), 7/60*(360/7)^(4/5), 
	1/20*(360/7)^(2/5)]
	[7/30*(360/7)^(3/5), 1, 1/5*(360/7)^(1/5), 7/120*(360/7)^(4/5), 7/5*(360/7)^(2/5), 
	3/5]
	[7/6*(360/7)^(2/5), 7/120*(360/7)^(4/5), 1, 7/24*(360/7)^(3/5), 7*(360/7)^(1/5), 
	7/120*(360/7)^(4/5)]
	[7/90*(360/7)^(4/5), 1/5*(360/7)^(1/5), 1/15*(360/7)^(2/5), 1, 7/15*(360/7)^(3/5), 
	1/5*(360/7)^(1/5)]
	[1/6*(360/7)^(1/5), 1/120*(360/7)^(3/5), 1/360*(360/7)^(4/5), 1/24*(360/7)^(2/5), 1, 
	1/120*(360/7)^(3/5)]
	[7/18*(360/7)^(3/5), 1, 1/3*(360/7)^(1/5), 7/72*(360/7)^(4/5), 7/3*(360/7)^(2/5), 
	1]
\end{lstlisting}

Вычисление $\delta$:
\begin{lstlisting}[language=c++,basicstyle=\footnotesize\ttfamily]
	delta = 7*(360/7)^(1/5)
	delta^-1 = 1/360*(360/7)^(4/5)
\end{lstlisting}
Вычисление наихудшего дифференцирующего вектора альтернатив:
\begin{lstlisting}[language=c++,basicstyle=\scriptsize\ttfamily]
Linearly independent w1 =
[1/20*(360/7)^(2/5), 1/20*(360/7)^(2/5)]
 [1, 3/5]
 [7/120*(360/7)^(4/5), 7/120*(360/7)^(4/5)]
 [1/5*(360/7)^(1/5), 1/5*(360/7)^(1/5)]
 [1/120*(360/7)^(3/5), 1/120*(360/7)^(3/5)]
 [1, 1]

Worst differentiating weight vector =
[1/60*(360/7)^(3/5)]
 [1/3*(360/7)^(1/5)]
 [1]
 [1/15*(360/7)^(2/5)]
 [1/360*(360/7)^(4/5)]
 [1/3*(360/7)^(1/5)]

D1 = [1, 7/3*(360/7)^(1/5), 3*(360/7)^(1/5)]
 [7, 1, 7/3*(360/7)^(1/5)]
 [9, 7, 1]
nu_1 = (360/7)^(1/10)*sqrt(27)

Worst differentiating vector of alternatives
x1 =
[1/9*(360/7)^(1/10)*sqrt(27)]
 [1]
 [1]
\end{lstlisting}
Вычисление наилучшего дифференцирующего вектора альтернатив:
\begin{lstlisting}[language=c++,basicstyle=\footnotesize\ttfamily]
Computing the best differenting weight vector
P =
 [1/20*(360/7)^(2/5), 1/20*(360/7)^(2/5)]
 [1, 3/5]
 [7/120*(360/7)^(4/5), 7/120*(360/7)^(4/5)]
 [1/5*(360/7)^(1/5), 1/5*(360/7)^(1/5)]
 [1/120*(360/7)^(3/5), 1/120*(360/7)^(3/5)]
 [1, 1]

Computing w_2

Linearly independent w2 =
 [1/20*(360/7)^(2/5), 1/20*(360/7)^(2/5)]
 [1, 3/5]
 [7/120*(360/7)^(4/5), 7/120*(360/7)^(4/5)]
 [1/5*(360/7)^(1/5), 1/5*(360/7)^(1/5)]
 [1/120*(360/7)^(3/5), 1/120*(360/7)^(3/5)]
 [1, 1]

Computing the best differentiating vector of alternatives
D2 =
 [7/120*(360/7)^(4/5), 7, 9]
 [49/120*(360/7)^(4/5), 7/120*(360/7)^(4/5), 7]
 [21/40*(360/7)^(4/5), 49/120*(360/7)^(4/5), 7/120*(360/7)^(4/5)]
nu_2 = (360/7)^(2/5)*sqrt(189/40)

Best differentiang vector of alternatives
x2 =
 [1/27*(360/7)^(3/5)*sqrt(189/40), 1/27*(360/7)^(3/5)*sqrt(189/40)]
 [1/21*(360/7)^(3/5)*sqrt(189/40), 7/9]
 [1, 1]
\end{lstlisting}

\end{document}

