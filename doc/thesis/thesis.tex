% !TeX spellcheck = ru_RU
\documentclass[specialist,
	substylefile = spbu.rtx,
	subf,href,colorlinks=true, 12pt]{disser}

\usepackage[utf8]{inputenc}
\usepackage[T2A]{fontenc}
\usepackage[english, russian]{babel}
\usepackage[a4paper,
  mag=1000,
  includefoot,
  left=3cm,
  right=1.5cm,
  top=2cm,
  bottom=2cm,
  headsep=1cm,
  footskip=1cm]{geometry}
\usepackage{longtable}
\usepackage{subcaption}
\usepackage{fancyvrb}
\usepackage{float}
\usepackage{listings}
\lstset{extendedchars=\true}
\usepackage{xcolor}
\usepackage{amsmath, amsfonts, amsthm}
\usepackage{graphics}
\usepackage{csquotes}
\usepackage{nicefrac}
\usepackage{minted}
\usepackage[
  backend=biber,
  bibstyle=gost-numeric,
  citestyle=gost-numeric,
  hyperref=true,
  natbib=true,
  url=true,
]{biblatex}

\usepackage{mathtools}
\usepackage{romannum}
\hypersetup{unicode=true}
\addbibresource{./ref.bib}
\let\vec=\mathbf
\setcounter{tocdepth}{2}
\author{Илья Шешуков}
\date{\number\year}
\title{Отчёт по научно-исследовательской работе}
\hypersetup{
 pdfauthor={Илья Шешуков},
 pdftitle={Курсовая работа},
 pdfkeywords={},
 pdfsubject={},
 pdfcreator={Emacs 26.3},
 pdflang={Russian}}

\newtheorem{utv}{Утверждение}

% Имена без курсива
\renewcommand*{\mkgostheading}[1]{#1}
% Правка записей типа thesis, чтобы дважды не писался автор
\DeclareBibliographyDriver{thesis}{
  \usebibmacro{bibindex}%
  \usebibmacro{begentry}%
  \usebibmacro{heading}%
  \newunit
  \usebibmacro{author}%
  \setunit*{\labelnamepunct}%
  \usebibmacro{thesistitle}%
  \setunit{\respdelim}%
  %\printnames[last-first:full]{author}%Вот эту строчку нужно убрать, чтобы автор диссертации не дублировался
  \newunit\newblock
  \printlist[semicolondelim]{specdata}%
  \newunit
  \usebibmacro{institution+location+date}%
  \newunit\newblock
  \usebibmacro{chapter+pages}%
  \newunit
  \printfield{pagetotal}%
  \newunit\newblock
  \usebibmacro{doi+eprint+url+note}%
  \newunit\newblock
  \usebibmacro{addendum+pubstate}%
  \setunit{\bibpagerefpunct}\newblock
  \usebibmacro{pageref}%
  \newunit\newblock
  \usebibmacro{related:init}%
  \usebibmacro{related}%
  \usebibmacro{finentry}
}

\theoremstyle{definition}
\newtheorem{definition}{Определение}[section]


\institution{
    Санкт-Петербургский государственный университет \\
    Прикладная математика и информатика \\
    Вычислительная стохастика и статистические модели
}

\topic{\normalfont\scshape%
Разработка вычислительных алгоритмов и программных средств поддержки принятия решений
}

% Научный руководитель
\sa       {Н.\,К.~Кривулин}
\sastatus {доктор физ.-мат. наук, профессор}

% Город и год
\city{Санкт-Петербург}


\begin{document}
\pagenumbering{arabic}

\title{Выпускная квалификационная работа}
\author{Шешуков Илья Вячеславович}
\institution{
    Санкт-Петербургский государственный университет \\
    Прикладная математика и информатика \\
    Вычислительная стохастика и статистические модели
}

\apname{д.\,ф.-м.\,н., профессор С.\,М.~Ермаков}

\topic{\normalfont\scshape%
Разработка вычислительных алгоритмов и программных средств поддержки принятия решений
}

% Научный руководитель
\sa       {Н.\,К.~Кривулин}
\sastatus {д.\,ф.-м.\,н., профессор}

% Рецензент
\rev{В.\,Н.~Сорокин}
\revstatus{к.\,ф.-м.\,н., ведущий программист}

% Город и год
\city{Санкт-Петербург}
\maketitle
\title{Graduation Thesis}
\author{Sheshukov Ilia Viacheslavovich}
\institution{
    Saint Petersburg State University
\\
    Faculty of Mathematics and Mechanics
\\
    Department of Statistical Modelling
}

\topic{\normalfont\scshape%
Development of computational algorithms and software tools for multi-criteria decision making
}

\apname{Professor S.\,M.~Ermakov}

% Научный руководитель
\sa       {N.\,K.~Krivulin}
\sastatus {Professor}

% Рецензент
\rev{V.\,N.~Sorokin}
\revstatus{Ph.\,D., Lead Programmer}

% Город и год
\city{Saint Petersburg}

\maketitle[en]
\tableofcontents
\intro

Задача принятия решения на основе парных сравнений состоит в том, чтобы по результатам попарных сравнений $n$ альтернатив определить абсолютный приоритет каждой из них. Подобные задачи возникают во множестве областей человеческой деятельности, где необходимо принимать решения на основе нескольких факторов.

Одним из методов решения подобных задач является метод log-чебышёвской аппроксимации. Преимущество этого метода в том, что решения получают в аналитическом виде, а не в численном. Однако, пока не существует общедоступных программных средств, которые бы решали данную задачу на компьютере.

В данной работе будет рассмотрен подход \cite{krivulin2019} в решении задачи оценки альтернатив на основе log-чебышёвской аппроксимации и представлено программное средство, решающее задачу многокритериальных парных сравнений данным методом.

Данная работа организована следующим образом: в Главе 1 и 2 рассматривается постановка задачи, в Главе 3 представлено её решение и разработанное программное средство.

\chapter{Задача оценки альтернатив на основе парных сравнений}
\section{Общая информация}
Задача принятия решений в самой общей формулировке состоит в следующем: из множества вариантов при заданных ограничениях нужно выбрать один или несколько вариантов, обеспечивающих решение некоторой задачи. 
Подобные задачи могут возникнуть в любой области, где человеку необходимо принять оптимальное решение на основе множества факторов: маркетинге, психологии, менеджменте и других.

Конкретной задачей принятия решений является задача парных сравнений. Её можно сформулировать так: имея $n$ альтернатив и результаты сравнения каждой пары (альтернатива $i$ \enquote{лучше} альтернативы $j$ в $m$ раз), необходимо упорядочить альтернативы. Также существует обобщение этой задачи на случай, когда альтернативы сравниваются по нескольким критериям, называемое \textit{многокритериальной задачей парных сравнений}.

Обычно такие сравнения проводят люди (эксперты, респонденты опросов), поэтому результаты парных сравнений в большинстве случаев нельзя наивно упорядочить. Рассмотрим две ситуации на Рис. \ref{pic:gr}.
\begin{figure*}[h]
	\centering
	\begin{subfigure}[t]{0.5\textwidth}
	  \centering

  		\begin{dot2tex}[tikz,options=-t math] 
  	digraph G {
  		T1[label="T_1"];
  		T2[label="T_2"];
  		T3[label="T_3"];
  		
  		T1 -> T2;
  		T3 -> T2;
  		T3 -> T1;
  	}
  \end{dot2tex}
	   \caption{Ситуация 1}
	\end{subfigure}%
	~ 
	\begin{subfigure}[t]{0.5\textwidth}
		\centering
      \begin{dot2tex}[tikz,options=-t math]
	digraph G {
		T1[label="T_1"];
		T2[label="T_2"];
		T3[label="T_3"];
		
		T1 -> T2;
		T2 -> T3;
		T3 -> T1;
	}
\end{dot2tex}
		\caption{Ситуация 2}
	\end{subfigure}
	\caption{Примеры возможных результатов парных сравнений. $T_i \rightarrow T_j$ означает	\enquote{$T_i$ лучше $T_j$} \label{pic:gr}}
\end{figure*}
В первой ситуации, упорядочить альтернативы по возрастанию приоритета просто: $T_2$, $T_1$, $T_3$. Во второй же ситуации при помощи такого наивного метода это не представляется возможным из-за цикла.

Существует множество методов решения данной задачи. Например, известны метод анализа иерархий Саати \cite{saaty_analytic_1980} и метод геометрических средних. Недостатком обоих методов является то, что решение получают численно. Кроме того, метод Саати является эвристическим.

В данной работе будет рассмотрен подход \cite{krivulin2019} в решении задачи оценки альтернатив на основе log-чебышёвской аппроксимации.
Преимущество рассматриваемого подхода в том, что решение получается аналитически, а не численно; недостатком является то, что полученное решение может быть не единственно.
\newpage
Целью работы является создание программного средства, решающего задачу многокритериальных парных сравнений данным методом.

\section{Постановка задачи}
\begin{definition}{\textit{Матрицей парных сравнений}}
	называется положительная матрица $\mathbf{A} = (a_{ij})$, элемент $a_{ij}$ которой показывает, во сколько раз альтернатива $i$ превосходит альтернативу $j$.
\end{definition}
\begin{remark}
	Матрицы парных сравнений являются \textit{кососимметрическими}. То есть, для матрицы $\mathbf{A} = (a_{ij})$ выполняется
	\[
	a_{ij} = \nicefrac{1}{a_{ji}}, \quad i,j = 1,\dots,n.
	\]
\end{remark}
\begin{definition}{\textit{Согласованной матрицей}}
	называется матрица $\mathbf{A} = (a_{ij})$ такая, что выполняется свойство транзитивности. Т.~е.
	\[
		a_{ij} = a_{ik}a_{kj}, \quad i,j,k = 1,\dots,n.
	\]
\end{definition}

\begin{utv}
	Необходимым и достаточным условием~\cite{saaty1984} того, что матрица $\mathbf{A}$
	согласована является существование вектора $\mathbf{x}$, называемого \textit{вектором абсолютных приоритетов}, такого, что
	\[
		a_{ij} = x_{i}/x_{j} \quad \forall i, j.
	\]
\end{utv}
На практике, матрицы парных сравнений обычно не согласованы, и поэтому возникает задача приближения матрицы парных сравнений согласованной матрицей.
Одним из способов решения (т.~н. $\log$-чебышёвская аппроксимация~\cite{krivulin2019}) этой задачи является нахождение вектора $\mathbf{x}$, который бы минимизировал следующее выражение
\[
	l_{\infty}\left(\mathbf{A}, \mathbf{x} \mathbf{x}^{-}\right)=\max _{1 \leq i, j \leq n}\left|\log a_{i j}-\log \frac{x_{i}}{x_{j}}\right|, \text{ где } \mathbf{x}^{-} = \begin{pmatrix}x_1^{-1}&x_2^{-1} & \dots & x_n^{-1}\end{pmatrix},
\]
Из монотонности логарифма и тождества $|x| = \max(x, -x)$ следует

\[
\max _{1 \leq i, j \leq n}\left|\log a_{i j}-\log \frac{x_{i}}{x_{j}}\right| = \log \max_{1 \leq i, j \leq n} \max\left(\frac{a_{ij}x_j}{x_i}, \frac{x_i}{a_{ij}x_j}\right),
\]
а из условия $a_{ij} = \nicefrac{1}{a_{ji}}$
\[
\max_{1 \leq i, j \leq n} \max\left(\frac{a_{ij}x_j}{x_i}, \frac{x_i}{a_{ij}x_j}\right) = \max_{1 \leq i, j \leq n} \frac{a_{ij}x_j}{x_i}.
\]
Исходная задача сводится \cite{krivulin2017} к нахождению $\mathbf{x}$ такого, что
\[
	\min _{\mathbf{x}>0} \max _{1 \leq i, j \leq n} \frac{a_{i j} x_{j}}{x_{i}}.
\]
Данная задача называется \textit{однокритериальной}. Также существует её обобщение на случай, когда сравнение происходит не по одному критерию, а сразу по нескольким.
Пусть $n$ альтернатив сравниваются по $m$ критериям.
Матрица $\mathbf{A}_k$ --- матрица попарных сравнений по критерию с номером $k = 1, \dots, n$.
Матрица $\mathbf{C} = (c_{ij})$ --- матрица, показывающая, во сколько раз критерий с номером $i$ важнее критерия с номером $j$. Необходимо построить вектор абсолютных приоритетов по матрицам $\mathbf{A}_k$ и $\mathbf{C}$.
Определим функцию
$$
f_{k}(\mathbf{x})=l_{\infty}\left(\mathbf{A}_{k}, \mathbf{x} \mathbf{x}^{-}\right)
$$ и тогда задача сводится к нахождению
$$
\min _{\mathbf{x}>\mathbf{0}}\left(f_{1}(\mathbf{x}), \ldots, f_{m}(\mathbf{x})\right).
$$

Эти задачи могут быть решены с использованием идемпотентной алгебры и max-$\times$ алгебры в частности.

Как уже было сказано ранее, данный метод не гарантирует единственность решения. С одной стороны, такой результат сложнее интерпретировать; с другой, при помощи введения дополнительных объектов --- \textit{наилучшего и наихудшего дифференцирующего вектора}, можно получить дополнительную информацию о множестве решений. Данные векторы будут рассматриваться в дальнейших главах.

\section{Max-$\times$ алгебра}

\begin{definition}
	\textit{Max-$\times$ алгебра}
	это алгебра над множеством $\mathbb{R}_{+} = \{x \in \mathbb{R} \mid x \geq 0\}$ с операциями ${(\oplus, \times)}$, где $\oplus$ --- это максимум, а $\times$ --- стандартное умножение.
\end{definition}

\section{Некоторые свойства max-$\times$ алгебры}
\begin{enumerate}
	\item Сложение идемпотентно
	\[
	x \oplus x = x \quad \forall x \in \mathbb{R}_+.
	\]
	\item Вычитание не определено.
	\item Экстремальное свойство сложения
	\[
	x \leq x \oplus y,\quad y \leq x \oplus y \quad \forall x, y \in \mathbb{R}_+.
	\]
	\item Изотонность сложения и умножения
	\[
	x \leq y \Rightarrow x \oplus z \leq y \oplus z,\quad xz \leq yz \quad \forall x,y,z \in \mathbb{R}_+.
	\]
	\item Антитонность обращения
	\[
	x \leq y \Rightarrow x^{-1} \geq y^{-1} \quad\forall x,y \in \mathbb{R}_+ \setminus \{0\}.
	\]
	\item Эквивалентность неравенств
	\[
	x \oplus y \leq z \Leftrightarrow x \leq z,\quad y \leq z \quad \forall x,y,z \in \mathbb{R}_+.
	\]
\end{enumerate}
\section{Векторная max-$\times$ алгебра}
Операции max-$\times$ можно естественным образом расширить на векторы и матрицы.

\begin{enumerate}
	\item Пусть $\textbf{A} = (a_{ij})$ и $\textbf{B} = (b_{ij})$ --- матрицы размерности $m \times n$. Операции сложения и умножения на скаляр $x$ определяются следующим образом
	\begin{align*}
		\{\textbf{A}\oplus\textbf{B}\}_{ij} &= a_{ik} \oplus b_{kj}, \\
		\{x\times\textbf{A}\}_{ij} &= xa_{ij}.
	\end{align*}
	\item Пусть $\textbf{A} = (a_{ij})$ --- матрица размерности $m \times n$ и $\textbf{B} = (b_{ij})$ --- матрица размерности $n \times k$. В max-$\times$ алгебре их произведение $\textbf{A}\times\textbf{B}$ определено следующим образом
	\[
	  \{\textbf{A}\times\textbf{B}\}_{ij} = \bigoplus_{k=1}^n a_{ik}b_{kj}.
	\]
\end{enumerate}

В дальнейшем будут использоваться понятия, определения которых приведём здесь.

\begin{definition}
Система векторов $(\textbf{a}_1, \textbf{a}_2, \dots, \textbf{a}_n)$ называется \textit{линейно независимой} тогда и только тогда, когда 
$$\delta(\textbf{A}) = \min _{1 \leq i \leq n} \Delta\left(\mathbf{A}_{(i)}, \mathbf{a}_{i}\right) > 1,$$
где $\mathbf{A}_{(i)}$ --- матрица $\mathbf{A}$ без $i$-го столбца, $\Delta(\mathbf{A}, \mathbf{b}) \coloneqq \left(\mathbf{A}\left(\mathbf{b}^{-} \mathbf{A}\right)^{-}\right)^{-} \mathbf{b}$, при условии $\textbf{b} > 0$ --- функция невязки.
\end{definition}

\begin{definition}
	Пусть $\textbf{A} = (a_{ij})$ --- матрица размерности $n \times n$. \textit{Следом} матрицы $\textbf{A}$ называется величина
	\[
	\tr \mathbf{A} \coloneqq \bigoplus_{i=1}^n a_{ii}.
	\]
\end{definition}

\begin{definition}{\textit{Тропический определитель} матрицы}
	\[
	\Tr \mathbf{A} \coloneqq \tr\mathbf{A} \oplus \dots \oplus \tr\mathbf{A}^n.
	\]

\end{definition}
\begin{definition}{\textit{Спектральный радиус} матрицы}
	\[
	\lambda \coloneqq \tr\mathbf{A} \oplus \dots \oplus \tr^{1/n}\mathbf{A}^n.
	\]
	Степень числа вычисляется стандартным образом, как обратный элемент по умножению.
\end{definition}
\begin{definition}{Оператор Клини\\}
	Если $\Tr \mathbf{A} \leq 1$, то \textit{оператор Клини} определяется следующим образом
	\[
	\mathbf{A}^* \coloneqq \mathbf{I} \oplus \mathbf{A} \oplus \dots \oplus \mathbf{A}^{n-1}.
	\]
\end{definition}

\section{Представление однокритериальной задачи в max-$\times$ алгебре}
На языке max-$\times$ алгебры задача будет выглядеть следующим образом
\[
	\min _{\mathbf{x}>0} \bigoplus_{1 \leq i, j \leq n} x_{i}^{-1} a_{i j} x_{j},
\]
что в векторной записи эквивалентно решению задачи
\[
	\min _{\mathbf{x}>0} \textbf{x}^{-} \textbf{A} \textbf{x}.
\]




\chapter{Многокритериальная задача парных сравнений}
%Программное средство, разрабатываемое в рамках этой работы, поддерживает и многокритериальные задачи.

\section{Постановка многокритериальной задачи}



В терминах max-$\times$ алгебры задача принимает вид
$$
	\min _{\mathbf{x}>\mathbf{0}} \mathbf{x}^{-} \mathbf{A} \mathbf{x} \quad \text { при } \quad \mathbf{A}=\bigoplus_{1 \leq k \leq m} w_{k} \mathbf{A}_{k},
$$
где вектор весов $\mathbf{w}=\left(w_{k}\right)$ находится как решение задачи
\[
	\min _{\mathbf{w}>0} \mathbf{w}^{-} \mathbf{C} \mathbf{w}.
\]
Все решения задачи определения весов записываются в виде
\[
	\mathbf{w} = (\lambda^{-1}\mathbf{C})^*\mathbf{v}, ~\mathbf{v} > 0,
\]
где $\lambda$ --- спектральный радиус матрицы $\mathbf{C}$. Аналогично, если $\mu$ --- спектральный радиус матрицы $\mathbf{A}$, то все решения задачи оценки рейтингов альтернатив имеют вид:
\[
	\mathbf{x} = (\mu^{-1}\mathbf{A})^*\mathbf{u},~\mathbf{u} > 0.
\]


\section{План решения многокритериальной задачи}
\begin{enumerate}
	\item По матрице $\mathbf{C}$ определяем вектор весов критериев:
	      \[
		      \mathbf{w} = (\mu^{-1}\mathbf{C})^*\mathbf{u},\quad  \mathbf{u} > 0,\quad \mu = \tr(\mathbf{C_1}) \oplus\ldots\oplus \tr^{1/m}(\mathbf{C_m}).
	      \]
	\item Если полученный вектор $\mathbf{w}$ не единственный, то находятся наихудший и наилучший дифференцирующий вектор весов:
	      \[
		      \mathbf{w}_1 = (\delta^{-1}\mathbf{11}^\mathrm{T} \oplus \mu^{-1}\mathbf{C})^*\mathbf{v_1},\quad  \mathbf{v_1} > 0,\quad \delta = \mathbf{1}^\mathrm{T}(\mu^{-}C)^*\mathbf{1},
	      \]
	      и
	      \[
		      \mathbf{w}_2 =\mathbf{P}(\mathbf{I}\oplus\mathbf{P}^{-}_{lk}\mathbf{P})\mathbf{v_2},\quad \mathbf{v_2} > 0,
	      \]
	      для которого матрица $\mathbf{P}$ получена из матрицы $(\mu^{-1}\mathbf{C})^*$ вычеркиванием линейно-зависимых столбцов, матрица $\mathbf{P}_{lk}$ --- из матрицы $\mathbf{P} = (p_{ij})$ заменой на ноль всех элементов, кроме $p_{lk}$, а индексы $k$ и $l$ определяются, исходя из условий
	      \[
		      k = \underset{j}{\arg\max}~\mathbf{1}^\mathrm{T}\mathbf{p}_j\mathbf{p}^-_j\mathbf{1},\quad l = \underset{i}{\arg\max} ~p^{-1}_{ik}.
	      \]
	\item С помощью векторов $\mathbf{w}_1 = (w^{(1)}_i)$ и $\mathbf{w}_2 = (w^{(2)}_i)$ составляются взвешенные
	      суммы матриц парных сравнений:
	      \begin{align*}
		      \mathbf{D}_1 & = w^{(1)}_1\mathbf{A_1}\oplus\ldots\oplus w^{(1)}_m\mathbf{A_m}, \\
		      \mathbf{D}_2 & =w^{(2)}_1\mathbf{A_1}\oplus\ldots\oplus w^{(2)}_m\mathbf{A_m}.
	      \end{align*}
	\item Вычисляется вектор рейтингов альтернатив для матрицы $\mathbf{D_1}$
	      \[
		      \mathbf{x} = (v^{-1}_1\mathbf{D_1})^*\mathbf{u}_1,\quad \mathbf{u}_1 > 0,\quad v_1 = \tr\mathbf{D}_1\oplus\dots\oplus \tr^{1/n}\mathbf{D}_1^n.
	      \]
	\item Если полученный вектор не единственный, то вместо него ищется наихудший дифференцирующий вектор
	      \[
		      \mathbf{x}_1 = (\delta^{-1}_1\mathbf{11}^\mathrm{T}\oplus v^{-1}_1\mathbf{D}_1)^*\mathbf{u_1},\quad \mathbf{u}_1 > 0,\quad \delta_1 = \mathbf{1}^\mathrm{T}(v^{-1}\mathbf{D_1})^*\mathbf{1}.
	      \]
	\item Вычисляется вектор рейтингов альтернатив для матрицы $\mathbf{D_2}$
	      \[
		      \mathbf{x}_2 = (v^{-1}_2\mathbf{D}_2)^*\mathbf{u}_2,\quad \mathbf{u_2} > 0,\quad v_2 = \tr\mathbf{D}_2\oplus\ldots\oplus\tr^{1/n}(\mathbf{D}^n_2).
	      \]

	\item Если этот вектор не единственный, то вместо него рассматриваем наилучший дифференцирующий вектор
	      \[
		      \mathbf{x}_2 =\mathbf{S}(\mathbf{I}\oplus\mathbf{S}^-_{lk}\mathbf{S})\mathbf{u}_2,~ \mathbf{u}_2 > 0,
	      \]

	      где матрица $\mathbf{S}$ получена из матрицы $(v^{-1}_2\mathbf{D}_2)^*$ вычеркиванием линейно-зависимых столбцов, матрица $\mathbf{S}^-_{lk}$ --- из матрицы $S = (s_{ij})$ обращением
	      в нуль всех элементов, кроме $s_{lk}$, а индексы $k$ и $l$ определяются, исходя из условий
	      \[
		      k = \underset{j}{\arg\max}~\mathbf{1}^\mathrm{T}\mathbf{s}_j\mathbf{s}^-_j\mathbf{1},~l = \underset{i}{\arg\max} ~s^-_{ik}.
	      \]
\end{enumerate}

\chapter{Разработка программных средств}
\section{Цель работы}
Требуется написать библиотеку и программу на языке \texttt{C++}, символьно решающую одно- и многокритериальную задачу принятия решений методом log-чебышёвской аппроксимации в max-алгебре.
\section{Выполнено}
\begin{enumerate}
	\item Было написано расширение для библиотеки линейной алгебры \texttt{Eigen} \cite{eigenweb} (версия 3.3.8), позволяющее работать с символьными вычислениями в max-алгебре.
	Для работы с символьными вычислениями использовалась библиотека \texttt{GiNaC} \cite{ginac_2006} (версия 1.8.0).
	\item Был реализован ряд функций, используемых в алгоритме решения задачи. Таких как
        нахождение тропического определителя, спектрального радиуса, оператор Клини, нахождение линейно независимых векторов и других.
	\item Задача одно- и многокритериального сравнения может быть полностью решена при помощи библиотеки.
\end{enumerate}

\section{Пример решения многокритериальной задачи}
\subsection{Условие}

\[
	\mathbf{C} = \begin{pmatrix}
		1 & 1/3 \\
		3 & 1
	\end{pmatrix}, \quad
	\mathbf{A}_1 = \begin{pmatrix}
		1   & 3 & 1/3 \\
		1/3 & 1 & 1   \\
		3   & 1 & 1
	\end{pmatrix}, \quad
	\mathbf{A}_2 =  \begin{pmatrix}
		1   & 1/3 & 5 \\
		3   & 1   & 7 \\
		1/5 & 1/7 & 1
	\end{pmatrix}.
\]
\subsection{Решение}
Найдём спектральный радиус матрицы \(\mathbf{C}\):

\[
	\lambda=\bigoplus_{k=1}^{2} \tr^{1 / k}\left(\mathbf{C}^{k}\right) = \tr\mathbf{C} \oplus \sqrt{\tr\mathbf{C}^2} =
	\tr\begin{pmatrix}1&1/3\\3&1\end{pmatrix} \oplus \sqrt{\tr\begin{pmatrix}1&1/3\\3&1\end{pmatrix}} = 1.
\]
В нашем случае, матрица Клини равна самой матрице \(\mathbf{C}\).
\[
	\mathbf{C}^* = \mathbf{I} \oplus \mathbf{C} = \mathbf{C}.
\]
Тогда
\[
	(\lambda^{-1}\mathbf{C})^*\mathbf{v} = (\lambda^{-1})\mathbf{C}\mathbf{v} = \mathbf{C}\mathbf{v}, \quad \mathbf{v} > 0.
\]
Столбцы матрицы коллинеарны, поэтому просто возьмём первый из них. Вектор весов критериев
\[
	\mathbf{w} = \begin{pmatrix}1 & 3\end{pmatrix}.
\]
Вычислим матрицу \(\mathbf{B}\):
\[
	\mathbf{B}=\bigoplus_{k=1}^{2} w_{k} \mathbf{A}_{k} = \begin{pmatrix}1&3&1/3\\1/3&1&1\\3&1&1\end{pmatrix} \oplus 3\begin{pmatrix}1&1/3&5\\3&1&7\\1/5&1/7&1\end{pmatrix} = \begin{pmatrix}3&3&15\\ 9&3&21\\ 3&1&3\\\end{pmatrix}.
\]
Спектральный радиус этой матрицы равен \(\mu = \sqrt{45}\).
Тогда вектор рейтингов альтернатив:
\begin{align*}
	\mathbf{x} = \mathbf{S}\mathbf{u} = \left(\frac{1}{\sqrt{45}}B\right)^*\mathbf{u} & =
	\mathbf{I} \oplus \frac{1}{\sqrt{45}} \mathbf{B} \oplus (\frac{1}{\sqrt{45}} \mathbf{B})^2 = \\
	&=\begin{pmatrix}1& 0& 0\\0& 1& 0\\0& 0& 1\end{pmatrix}
	\oplus
	\begin{pmatrix}
		1/\sqrt{5} & 1/\sqrt{5}     & \sqrt{5}   \\
		3/\sqrt{5} & 1/\sqrt{5}     & 7/\sqrt{5} \\
		1/\sqrt{5} & 1/(3 \sqrt{5}) & 1/\sqrt{5} \\
	\end{pmatrix}   \oplus                                      \\
	                                                           & \oplus
	\begin{pmatrix}1& 1/3& 7/5\\7/5& 3/5& 3\\1/5& 1/5& 1\end{pmatrix} =
	\begin{pmatrix} 1&1/\sqrt{5}&\sqrt{5}\\ 7/5&1&7/\sqrt{5}\\ 1/\sqrt{5}&1/5&1 \end{pmatrix}\mathbf{u}.
\end{align*}

Здесь только первый и второй столбцы являются линейно независимыми, значит будем искать наихудший и наилучший дифференцирующий вектор.

\subsubsection{Наихудший дифференцирующий вектор}
\[
	\delta = \mathbf{1}^\mathrm{T}(\mu^{-1}\mathbf{B})^*\mathbf{1} = 7/\sqrt{5}.
\]

\begin{align*}
	\mathbf{x_2} & = (\delta^{-1}\mathbf{1}\mathbf{1}^\mathrm{T} \oplus \mu^{-1}\mathbf{B})^*\mathbf{u} = \begin{pmatrix}\sqrt{45}/15 & \sqrt{45}/15 & \sqrt{45}/3\\ \sqrt{45}/5 & \sqrt{45}/15 & 7\sqrt{45}/15 \\ \sqrt{45}/15 & \sqrt{45}/45 & \sqrt{45}/15 \end{pmatrix}^*u =                   \\
	                 & = \mathbf{I} \oplus (\delta^{-1}\mathbf{1}\mathbf{1}^\mathrm{T} \oplus \mu^{-1}\mathbf{B}) \oplus (\delta^{-1}\mathbf{1}\mathbf{1}^\mathrm{T} \oplus \mu^{-1}\mathbf{B})^2 = \\
	                 & = \begin{pmatrix} 1&0&0 \\ 0&1&0 \\ 0&0&1 \end{pmatrix}
	\oplus
	\begin{pmatrix}1/\sqrt{5}&1/\sqrt{5}&\sqrt{5}\\3/\sqrt{5}&1/\sqrt{5}&7/\sqrt{5}\\1/\sqrt{5}&3\sqrt{5}/7&1/\sqrt{5}\end{pmatrix}
	\oplus
	\begin{pmatrix}1&5/7&7/5\\7/5&1&3\\3/7&1/5&1\end{pmatrix}             =                                                                        \\
	                 & = \begin{pmatrix} 1&5/7&\sqrt{5}\\ 7/5&1&7/\sqrt{5}\\ 1/\sqrt{5}&\sqrt{5}/7&1 \end{pmatrix}\mathbf{u}, \quad \mathbf{u} > 0.
\end{align*}

Тут все столбцы коллинеарны. Возьмём первый и нормируем:

\begin{align*}
	\begin{pmatrix}0.3512\\0.4917\\0.1570\end{pmatrix}.
\end{align*}

Порядок альтернатив: (\Romannum{2}) \(\succ\) (\Romannum{1}) \(\succ\) (\Romannum{3}).
\subsubsection{Наилучший дифференцирующий вектор}
\label{sec:org896b7a9}
Найденный нами наихудший дифференцирующий вектор, это первый столбец матрицы $\mathbf{S}$. Можем предположить, что наилучший дифференцирующий вектор окажется вторым столбцом этой матрицы. Давайте это проверим.

Из предыдущего получили, что \(\mathbf{S} = \begin{pmatrix}1&1/\sqrt{5}\\7/5&1\\1/\sqrt{5}&1/5\end{pmatrix}\).
Для нахождения наилучшего дифференцирующего вектора найдём число \(\Delta\):

\[
	\Delta = \mathbf{1}^\mathrm{T} \mathbf{S} \mathbf{S}^{-} \mathbf{1} = \begin{pmatrix}7/5&1\end{pmatrix}\begin{pmatrix}\sqrt{5}\\5\end{pmatrix} = 5.
\]
Все наилучшие дифференцирующие решения имеют вид

\[
	\mathbf{x}_{1}=\mathbf{S}\left(\mathbf{I} \oplus \mathbf{S}_{l k}^{-} \mathbf{S}\right) \mathbf{v}, \quad \mathbf{v}>\mathbf{0},
\]
где индексы \(k\) и \(l\) определяются из условия

\[
	\mathbf{1}^\mathrm{T} \mathbf{s}_{k} s_{l k}^{-1}=\Delta = 5.
\]
Это условие выполняется только при \(k=2\) и \(l=3\). Тогда наилучшие решения генерируют столбцы матрицы:

\begin{align*}
	\mathbf{x}_{1} & =\mathbf{S}\left(\mathbf{I} \oplus \mathbf{S}_{3,2}^{-} \mathbf{S}\right) \mathbf{v} = \\ 
	&= \begin{pmatrix}1&1/\sqrt{5}\\7/5&1\\1/\sqrt{5}&1/5\end{pmatrix} \left( \begin{pmatrix}1&0\\0&1\end{pmatrix} \oplus \begin{pmatrix}0&0&0\\0&0&5\end{pmatrix}\begin{pmatrix}1&  1/\sqrt{5}\\7/5&1\\1/\sqrt{5}&1/5\end{pmatrix} \right)\mathbf{v} = \\
	                   & = \begin{pmatrix}1&1/\sqrt{5}\\\sqrt{5}&1\\1/\sqrt{5}&1/5\end{pmatrix}\mathbf{v}.
\end{align*}
Столбцы этой матрицы коллинеарны, возьмём второй.

Как и предполагалось, наилучший дифференцирующий вектор коллинеарен второму столбцу матрицы \(\mathbf{S}\).

\[
	x_1 = \begin{pmatrix}
		1/\sqrt{5} \\1\\1/5
	\end{pmatrix} \approx \begin{pmatrix}
		0.4472 \\ 1 \\ 0.2
	\end{pmatrix}.
\]

Порядок альтернатив: (\Romannum{2}) \(\succ\) (\Romannum{1}) \(\succ\) (\Romannum{3}).

\textbf{Общий ответ}: (\Romannum{2}) \(\succ\) (\Romannum{1}) \(\succ\) (\Romannum{3}).

\subsection{Использование библиотеки для вычислений}
Приведём пример вывода программы для получения вышеприведённых вычислений на компьютере (более сложный и подробный пример может быть найден в Приложении \ref{appendix:full}).

\begin{verbatim}
Computing spectral radius: lambda = 1

Computing (1/lambda * C)^*
(1/lambda * C)^* =
[1, 1/3]
 [3, 1]

Computing delta:
delta = 3, delta^-1 = 1/3

Computing the worst differentiating weight vector
Linearly independent w1 =
[1/3]
 [1]

Worst differentiating weight vector =
[1/3]
 [1]

D1 = [1, 1, 5]
 [3, 1, 7]
 [1, 1/3, 1]
nu_1 = sqrt(5)

Worst differentiating vector of alternatives
x1 =
[sqrt(5)]
 [7/5*sqrt(5)]
 [1]


Computing the best differenting weight vector
P =
[1/3]
 [1]

Computing w_2

Linearly independent w2 =
[1/3]
 [1]

Computing the best differentiating vector of alternatives
D2 = [1, 1, 5]
 [3, 1, 7]
 [1, 1/3, 1]
nu_2 = sqrt(5)

Best differentiang vector of alternatives
x2 = [sqrt(5)]
 [5]
 [1]
\end{verbatim}

Получили следующий порядок альтернатив:
для наихудшего дифференцирующего вектора (\Romannum{2}) \(\succ\) (\Romannum{1}) \(\succ\) (\Romannum{3}),
для наилучшего --- (\Romannum{2}) \(\succ\) (\Romannum{1}) \(\succ\) (\Romannum{3}).

Результат совпадает с результатом, полученным ручными вычислениями.

\subsection{Внутреннее устройство библиотеки}
Библиотека представляет собой
\begin{enumerate}
	\item Класс \texttt{MaxAlgebra<T>} реализующий операции в max-алгебре.

\begin{lstlisting}[language=C++,basicstyle=\footnotesize\ttfamily,keywordstyle=\color{blue}]
Template<typename Op>
class MaxAlgebra {
 public:
  GiNaC::ex value;
  MaxAlgebra() : value(){};
  ...
  friend MaxAlgebra operator+(const MaxAlgebra& lhs, const MaxAlgebra& rhs);
  friend MaxAlgebra operator*(const MaxAlgebra& lhs, const MaxAlgebra& rhs);
  friend MaxAlgebra operator/(const MaxAlgebra& lhs, const MaxAlgebra& rhs);
  friend bool operator<(const MaxAlgebra& lhs, const MaxAlgebra& rhs);
  friend bool operator==(const MaxAlgebra& lhs, const MaxAlgebra& rhs);
  friend bool operator>(const MaxAlgebra& lhs, const MaxAlgebra& rhs);
  friend MaxAlgebra pow(const MaxAlgebra& lhs, const MaxAlgebra& rhs);
  friend std::ostream& operator<<(std::ostream& out, const MaxAlgebra& val);
  ...
  MaxAlgebra& operator=(const MaxAlgebra& rhs);
  MaxAlgebra& operator+=(const MaxAlgebra& rhs);
  MaxAlgebra& operator*=(const MaxAlgebra& rhs);
  MaxAlgebra& operator/=(const MaxAlgebra& rhs);
  MaxAlgebra& abs(const MaxAlgebra& rhs);
};
\end{lstlisting}
\end{enumerate}

\begin{enumerate}
 \setcounter{enumi}{1}
	\item Две специализации класса реализующие операции в max-+ и max-$\times$-алгебре.
\begin{lstlisting}[language=C++,basicstyle=\footnotesize\ttfamily,keywordstyle=\color{blue}]

using MaxTimes = MaxAlgebra<std::multiplies<void>>;
using MaxPlus = MaxAlgebra<std::plus<void>>;

MaxPlus operator*(const MaxPlus& lhs, const MaxPlus& rhs);
MaxPlus operator/(const MaxPlus& lhs, const MaxPlus& rhs);
...
MaxTimes operator*(const MaxTimes& lhs, const MaxTimes& rhs);
MaxTimes operator/(const MaxTimes& lhs, const MaxTimes& rhs);
\end{lstlisting}
\end{enumerate}

\begin{enumerate}
  \setcounter{enumi}{2}
		\item Расширение для библиотеки \texttt{Eigen}, позволяющее использовать \texttt{MaxAlgebra<T>} в качестве элементов матрицы.
		\item Набор функций, участвующих в алгоритме решения задачи: нахождение тропического определителя, спектрального радиуса, оператор Клини, нахождение линейно независимых векторов и т. д.
\end{enumerate}
\subsection{Список реализованных функций}

\begin{enumerate}
	\item Арифметические операции над числами в max-+ и max-$\times$-алгебре: $\oplus$, $\times$, возведение в степень и т. д.
	\item Операции над векторами и матрицами: перемножение, транспонирование, возведение в степень и т. д.
	\item Функции, необходимые в решении задачи: спектральный радиус, получение матрицы Клини, нахождение коэффициентов в задаче нахождение наилучшего дифференцирующего вектора.
\end{enumerate}

%BEGIN_FOLD
\conclusion
В ходе данной работы были изучены различные методы решения задач принятия решений в одно- и многокритериальной постановке.

На языке \texttt{C++} с применением библиотек \texttt{Eigen} и \texttt{GiNaC} было разработано программной средство (библиотека и программа), реализующее как математический аппарат max-$\times$ алгебры, так и метод log-чебышёвской аппроксимации. Полные программные коды, распространяемые под свободной лицензией, находятся на сайте \url{https://github.com/isheshukov/vkr-2021}.

Разработанное средство обеспечивает возможность полного решения задачи.
Результаты, полученные в результате работы программы совпадают с результатами, полученными ручным подсчётом по данному методу.

-
%END_FOLD
\addcontentsline{toc}{chapter}{Список литературы}
\printbibliography

\appendix
\cleardoublepage\makeatletter\@openrightfalse\makeatother
\chapter{Примеры работы программы}
\section{Пример 1}
Объявление двух матриц $A_{1}$ и $A_{2}$ размера $3\times 3$:
\begin{lstlisting}[language=C++,basicstyle=\ttfamily,keywordstyle=\color{red}]
 MatrixDn A1(3, 3);
 MatrixDn A2(3, 3);

 A1 << ex(sin(1)), 3, ex(1) / 3,
        ex(1) / 3, 1, 1,
                3, 1, 1;
 A2 <<   0, ex(1) / 3, 5,
         3,         1, 7,
 ex(1) / 5, ex(1) / 7, 1;
  \end{lstlisting}
Сумма и произведение этих матриц:
\begin{lstlisting}[language=C++,basicstyle=\ttfamily,keywordstyle=\color{red}]
 std::cout << "A1+A2=\n" << A1+A2 << std::endl;
 std::cout << "A1*A2=\n" << A1*A2 << std::endl;
  \end{lstlisting}
Вывод:
\begin{lstlisting}[language=bash,basicstyle=\ttfamily,keywordstyle=\color{red}]
 A1 + A2 =               A1 * A2 =
   \sin(1) 3 5             9  3 21
   3       1 7             3  1  7
   3       1 1             3  1 15
  \end{lstlisting}
\section{Пример 2}
Объявление матрицы $C$ размера $4\times 4$:
\begin{lstlisting}[language=bash,basicstyle=\ttfamily,keywordstyle=\color{red}]
   MatrixDn C(4, 4);
   C <<  8, 3, 10, 6,
         7, 2,  1, 4,
        10, 8,  2, 2,
         4, 7,  6, 1;
  \end{lstlisting}
Вычисление спектрального радиуса:
\begin{lstlisting}[language=bash,basicstyle=\ttfamily,keywordstyle=\color{red}]
   std::cout
         << "Spectral radius of C = "
         << spectral_radius(C)
         << std::endl;
   \end{lstlisting}
Вывод:
\begin{lstlisting}[language=bash,basicstyle=\ttfamily,keywordstyle=\color{red}]
m^1=                      m^2=
 8  3 10  6               100  80  80  48
 7  2  1  4                56  28  70  42
10  8  2  2                80  30 100  60
 4  7  6  1                60  48  40  28
                          tr(m^2)^(1/2) = 10

m^3=                      m^4=
 800  640 1000  600       10000  8000  8000  4800
 700  560  560  336        5600  4480  7000  4200
1000  800  800  480        8000  6400 10000  6000
 480  320  600  360        6000  4800  4800  2880
tr(m^3)^(1/3) = 800^(1/3)   tr(m^4)^(1/4) = 10

Spectral radius of C = 10
\end{lstlisting}
\section{Пример 3. Решение многокритериальной задачи парного сравнения} \label{appendix:full}

\subsubsection{Условие}
$$
	\begin{array}{l}
		\mathbf{C}=\left(\begin{array}{cccccc}
				1     & 1 / 4 & 1 / 5 & 1 / 4 & 6 & 1 / 6 \\
				4     & 1     & 1 / 3 & 3     & 6 & 1 / 2 \\
				5     & 3     & 1     & 4     & 7 & 3     \\
				4     & 1 / 3 & 1 / 4 & 1     & 5 & 1 / 5 \\
				1 / 6 & 1 / 6 & 1 / 7 & 1 / 5 & 1 & 1 / 7 \\
				6     & 2     & 1 / 3 & 5     & 7 & 1
			\end{array}\right)                                                                                                                \\
		\mathbf{A}_{1}=\left(\begin{array}{ccc}
				1     & 5     & 8 \\
				1 / 5 & 1     & 5 \\
				1 / 8 & 1 / 5 & 1
			\end{array}\right), \quad \mathbf{A}_{2}=\left(\begin{array}{ccc}
				1     & 7     & 9 \\
				1 / 7 & 1     & 7 \\
				1 / 9 & 1 / 7 & 1
			\end{array}\right), \quad \mathbf{A}_{3}=\left(\begin{array}{ccc}
				1 & 1 / 7 & 1 / 9 \\
				7 & 1     & 1 / 7 \\
				9 & 7     & 1
			\end{array}\right), \\
		\mathbf{A}_{4}=\left(\begin{array}{ccc}
				1     & 3     & 5 \\
				1 / 3 & 1     & 4 \\
				1 / 5 & 1 / 4 & 1
			\end{array}\right), \quad \mathbf{A}_{5}=\left(\begin{array}{ccc}
				1     & 3     & 5 \\
				1 / 3 & 1     & 4 \\
				1 / 5 & 1 / 4 & 1
			\end{array}\right), \quad \mathbf{A}_{6}=\left(\begin{array}{ccc}
				1     & 7     & 9 \\
				1 / 7 & 1     & 7 \\
				1 / 9 & 1 / 7 & 1
			\end{array}\right)
	\end{array}
$$

\subsubsection{Код программы}

Задание матриц:
\begin{lstlisting}[language=c++,basicstyle=\small\ttfamily,keywordstyle=\color{red}]
MatrixDn C(6, 6);
C << 1, MaxTimes(1, 4), MaxTimes(1, 5), MaxTimes(1, 4), 6,
    MaxTimes(1, 6), 4, 1, MaxTimes(1, 3), 3, 6, MaxTimes(1, 2), 5, 3, 1,
    4, 7, 3, 4, MaxTimes(1, 3), MaxTimes(1, 4), 1, 5, MaxTimes(1, 5),
    MaxTimes(1, 6), MaxTimes(1, 6), MaxTimes(1, 7), MaxTimes(1, 5), 1,
    MaxTimes(1, 7), 6, 2, MaxTimes(1, 3), 5, 7, 1;

std::vector<MatrixDn>As;
MatrixDn A1(3, 3);
A1 << 1, 5, 8, ma(1, 5), 1, 5, ma(1, 8), ma(1, 5), 1;
MatrixDn A2(3, 3);
A2 << 1, 7, 9, ma(1, 7), 1, 7, ma(1, 9), ma(1, 7), 1;
MatrixDn A3(3, 3);
A3 << 1, ma(1, 7), ma(1, 9), 7, 1, ma(1, 7), 9, 7, 1;
MatrixDn A4(3, 3);
A4 << 1, 3, 5, ma(1, 3), 1, 4, ma(1, 5), ma(1, 4), 1;
MatrixDn A5(3, 3);
A5 << 1, 3, 5, ma(1, 3), 1, 4, ma(1, 5), ma(1, 4), 1;
MatrixDn A6(3, 3);
A6 << 1, 7, 9, ma(1, 7), 1, 7, ma(1, 9), ma(1, 7), 1;

As.push_back(A1);
As.push_back(A2);
As.push_back(A3);
As.push_back(A4);
As.push_back(A5);
As.push_back(A6);

const auto num_crit_mat = C.rows();
const auto num_comp_mat = A1.rows();
\end{lstlisting}

Вычисление спектрального радиуса:
\begin{lstlisting}[language=c++,basicstyle=\footnotesize\ttfamily]
auto lambda = spectral_radius(C);
\end{lstlisting}

Вычисление $(\lambda^{-1}C)^*$:
\begin{lstlisting}[language=c++,basicstyle=\footnotesize\ttfamily]
MatrixDn Calmoststar = (1 / lambda) * C;
auto w = cleany(Calmoststar);
\end{lstlisting}


Вычисление $\delta$:
\begin{lstlisting}[language=c++,basicstyle=\footnotesize\ttfamily]
auto delta = (MatrixDn::Ones(1, num_crit_mat) * w * MatrixDn::Ones(num_crit_mat, 1)).value();
\end{lstlisting}

Вычисление наихудшего дифференцирующего вектора:
\begin{lstlisting}[language=c++,basicstyle=\footnotesize\ttfamily]
MatrixDn W1m = (1 / delta * MatrixDn::Ones(num_crit_mat, num_crit_mat)) + Calmoststar;
auto w1 = cleany(W1m);
auto liw1 = linearly_independent_system(w1);
\end{lstlisting}

Получилась матрица из двух линейно независимых векторов. Попробуем получить \enquote{самый наихудший} вектор. Возьмём линейнонезависимые столбцы матрицы, нормируем по максимуму и сложим:
\begin{lstlisting}[language=c++,basicstyle=\footnotesize\ttfamily]
MatrixDn worst_vector = MatrixDn::Zero(liw1.rows(), 1);
for (int i = 0; i < liw1.cols(); ++i){
    worst_vector += liw1.col(i) / liw1.col(i).maxCoeff();
}
\end{lstlisting}


Получим взвешенную сумму:
\begin{lstlisting}[language=c++,basicstyle=\footnotesize\ttfamily]
MatrixDn D1 = MatrixDn::Zero(num_comp_mat, num_comp_mat);
for (int i = 0; i < worst_vector.rows(); ++i){
    D1 += worst_vector(i) * As[i];
}

auto nu_1 = spectral_radius(D1);
MatrixDn almost_x1 = (1 / nu_1) * D1;
auto x1 = cleany(almost_x1);
auto LI_x1 = linearly_independent_system(x1);
\end{lstlisting}

Вычисляем вектор рейтингов альтернатив матрицы $\mathbf{D}_{1}$.
Если матрица в переменной \texttt{LI\_x1} состоит из одного столбца, то этот столбец является решением. В противном случае неединственности решения:

\begin{lstlisting}[language=c++,basicstyle=\footnotesize\ttfamily]
auto delta_1 = (MatrixDn::Ones(1, x1.rows()) *
  x1 *
  MatrixDn::Ones(x1.cols(), 1)).value();

MatrixDn almost_x1new = ((1/delta_1) *
  MatrixDn::Ones(x1.cols(), x1.rows())) +
  almost_x1;

auto x1new = cleany(almost_x1new);

auto LI_x1new = linearly_independent_system(x1new);
\end{lstlisting}

Вычисление наилучшего дифференцирующего вектора.
Матрица $P$ --- линейно-независимые столбцы матрицы $w$.
\begin{lstlisting}[language=c++,basicstyle=\footnotesize\ttfamily]
MatrixDn P = linearly_independent_system(w);
\end{lstlisting}

Получение коэффициентов $k$ и $l$:
\begin{lstlisting}[language=c++,basicstyle=\footnotesize\ttfamily]
auto bdfc = best_diff_vector_coefficients(P);
\end{lstlisting}


Вычисление $w_2$. На этом этапе вычислим $w_{2}$ для каждой пары коэфициентов из предыдущего шага, объединим в одну матрицу и найдём линейнонезависимую часть.
\begin{lstlisting}[language=c++,basicstyle=\footnotesize\ttfamily]
for (auto& [k, l]: bdfc){

    MatrixDn Plk_inverse = MatrixDn::Zero(P.rows(), P.cols());
    Plk_inverse(l, k) = 1 / P(l, k);
    auto t = P * (MatrixDn::Identity(P.cols(), P.cols()) +
                   Plk_inverse.transpose() * P);
    W2.conservativeResize(t.rows(), W2.cols() + t.cols());
    W2.rightCols(t.cols()) = t;
}
auto LI_W2 = linearly_independent_system(W2);
auto nu_2 = spectral_radius(D2);
MatrixDn almost_x2 = (1 / nu_2) * D2;

auto x2 = cleany(almost_x2);
auto LI_x2 = linearly_independent_system(x2);
\end{lstlisting}

Вычислияем вектор рейтингов альтернатив для $\mathbf{D}_2$.
Если матрица в переменной \texttt{LI\_x2} состоит из одного столбца, то этот столбец является решением. В противном случае неединственности решения:

\begin{lstlisting}[language=c++,basicstyle=\footnotesize\ttfamily]
auto S = LI_x2;
auto bdfc = best_diff_vector_coefficients(S);
MatrixDn WS;
for (auto& [k, l]: bdfc){
    MatrixDn Slk_inverse = MatrixDn::Zero(S.rows(), S.cols());
    Slk_inverse(l, k) = 1 / S(l, k);
    auto t = S * (MatrixDn::Identity(S.cols(), S.cols()) +
                   Slk_inverse.transpose() * S);
    WS.conservativeResize(t.rows(), WS.cols() + t.cols());
    WS.rightCols(t.cols()) = t;
}
auto LI_WS = linearly_independent_system(WS);
\end{lstlisting}

\subsubsection{Вывод программы}

Вычисление  спектрального радиуса:
\begin{lstlisting}[language=c++,basicstyle=\footnotesize\ttfamily]
	lambda = (360/7)^(1/5)
\end{lstlisting}

Вычисление $(\lambda^{-1}C)^*$:
\begin{lstlisting}[language=c++,basicstyle=\scriptsize\ttfamily]
	w = 
	[1, 1/20*(360/7)^(2/5), 1/60*(360/7)^(3/5), 1/4*(360/7)^(1/5), 7/60*(360/7)^(4/5), 
	1/20*(360/7)^(2/5)]
	[7/30*(360/7)^(3/5), 1, 1/5*(360/7)^(1/5), 7/120*(360/7)^(4/5), 7/5*(360/7)^(2/5), 
	3/5]
	[7/6*(360/7)^(2/5), 7/120*(360/7)^(4/5), 1, 7/24*(360/7)^(3/5), 7*(360/7)^(1/5), 
	7/120*(360/7)^(4/5)]
	[7/90*(360/7)^(4/5), 1/5*(360/7)^(1/5), 1/15*(360/7)^(2/5), 1, 7/15*(360/7)^(3/5), 
	1/5*(360/7)^(1/5)]
	[1/6*(360/7)^(1/5), 1/120*(360/7)^(3/5), 1/360*(360/7)^(4/5), 1/24*(360/7)^(2/5), 1, 
	1/120*(360/7)^(3/5)]
	[7/18*(360/7)^(3/5), 1, 1/3*(360/7)^(1/5), 7/72*(360/7)^(4/5), 7/3*(360/7)^(2/5), 
	1]
\end{lstlisting}

Вычисление $\delta$:
\begin{lstlisting}[language=c++,basicstyle=\footnotesize\ttfamily]
	delta = 7*(360/7)^(1/5)
	delta^-1 = 1/360*(360/7)^(4/5)
\end{lstlisting}
Вычисление наихудшего дифференцирующего вектора альтернатив:
\begin{lstlisting}[language=c++,basicstyle=\scriptsize\ttfamily]
Linearly independent w1 =
[1/20*(360/7)^(2/5), 1/20*(360/7)^(2/5)]
 [1, 3/5]
 [7/120*(360/7)^(4/5), 7/120*(360/7)^(4/5)]
 [1/5*(360/7)^(1/5), 1/5*(360/7)^(1/5)]
 [1/120*(360/7)^(3/5), 1/120*(360/7)^(3/5)]
 [1, 1]

Worst differentiating weight vector =
[1/60*(360/7)^(3/5)]
 [1/3*(360/7)^(1/5)]
 [1]
 [1/15*(360/7)^(2/5)]
 [1/360*(360/7)^(4/5)]
 [1/3*(360/7)^(1/5)]

D1 = [1, 7/3*(360/7)^(1/5), 3*(360/7)^(1/5)]
 [7, 1, 7/3*(360/7)^(1/5)]
 [9, 7, 1]
nu_1 = (360/7)^(1/10)*sqrt(27)

Worst differentiating vector of alternatives
x1 =
[1/9*(360/7)^(1/10)*sqrt(27)]
 [1]
 [1]
\end{lstlisting}
Вычисление наилучшего дифференцирующего вектора альтернатив:
\begin{lstlisting}[language=c++,basicstyle=\footnotesize\ttfamily]
Computing the best differenting weight vector
P =
 [1/20*(360/7)^(2/5), 1/20*(360/7)^(2/5)]
 [1, 3/5]
 [7/120*(360/7)^(4/5), 7/120*(360/7)^(4/5)]
 [1/5*(360/7)^(1/5), 1/5*(360/7)^(1/5)]
 [1/120*(360/7)^(3/5), 1/120*(360/7)^(3/5)]
 [1, 1]

Computing w_2

Linearly independent w2 =
 [1/20*(360/7)^(2/5), 1/20*(360/7)^(2/5)]
 [1, 3/5]
 [7/120*(360/7)^(4/5), 7/120*(360/7)^(4/5)]
 [1/5*(360/7)^(1/5), 1/5*(360/7)^(1/5)]
 [1/120*(360/7)^(3/5), 1/120*(360/7)^(3/5)]
 [1, 1]

Computing the best differentiating vector of alternatives
D2 =
 [7/120*(360/7)^(4/5), 7, 9]
 [49/120*(360/7)^(4/5), 7/120*(360/7)^(4/5), 7]
 [21/40*(360/7)^(4/5), 49/120*(360/7)^(4/5), 7/120*(360/7)^(4/5)]
nu_2 = (360/7)^(2/5)*sqrt(189/40)

Best differentiang vector of alternatives
x2 =
 [1/27*(360/7)^(3/5)*sqrt(189/40), 1/27*(360/7)^(3/5)*sqrt(189/40)]
 [1/21*(360/7)^(3/5)*sqrt(189/40), 7/9]
 [1, 1]
\end{lstlisting}

\end{document}

