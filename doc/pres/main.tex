\documentclass[11pt]{beamer}
\usepackage[utf8]{inputenc}
\usepackage[T2A]{fontenc}
\usepackage[english, russian]{babel}
\usepackage{nicefrac}
\usepackage{mathtools, amsmath, amsfonts}
\usepackage{listings}
\usepackage{bm}

\usepackage{csquotes}


\usepackage[
backend=biber, 
sorting=nyt,
bibstyle=gost-authoryear,
citestyle=gost-authoryear
]{biblatex}

\addbibresource{./ref.bib}

\usetheme{Madrid}

\makeatletter
\defbeamertemplate*{footline}{infolineslong}
{
  \leavevmode%
  \hbox{%
  \begin{beamercolorbox}[wd=.3\paperwidth,ht=2.5ex,dp=1ex,center]{author in head/foot}%
    \usebeamerfont{author in head/foot}\insertshortauthor\expandafter\beamer@ifempty\expandafter{\beamer@shortinstitute}{}{~~(\insertshortinstitute)}
  \end{beamercolorbox}%
  \begin{beamercolorbox}[wd=.55\paperwidth,ht=2.5ex,dp=1ex,center]{title in head/foot}%
    \usebeamerfont{title in head/foot}\insertshorttitle[width=\textwidth, respectlinebreaks]
  \end{beamercolorbox}%
  \begin{beamercolorbox}[wd=.15\paperwidth,ht=2.5ex,dp=1ex,right]{date in head/foot}%
%    \usebeamerfont{date in head/foot}\insertshortdate{}\hspace*{2em}
\insertframenumber{} / \inserttotalframenumber\hspace*{2ex} 
  \end{beamercolorbox}}%
  \vskip0pt%
}

\uselanguage{Russian}
\languagepath{Russian}


\makeatother
\title[Разработка алгоритмов и ПО поддержки принятия решений]{Разработка вычислительных алгоритмов и программных средств поддержки принятия решений}
\author{Шешуков Илья Вячеславович}
\institute[]{
Санкт-Петербургский государственный университет \\
Прикладная математика и информатика \\
Кафедра статистического моделирования \\
\textbf{Научный руководитель:}  доктор физ.-мат. наук, профессор Н.\,К.~Кривулин
}

\begin{document}

\begin{frame}
\titlepage
\end{frame}

\begin{frame}{Задача принятия решений}
   \begin{itemize}
     \item Имеется некоторая задача, для решения которой есть $n$ альтернатив (возможных решений).
     \item Также имеются результаты сравнений каждой пары альтернатив между собой.
     \item Требуется на основе парных сравнений определить абсолютный приоритет каждой альтернативы.
   \end{itemize}

\end{frame}
\begin{frame}{Согласованная матрица парных сравнений}
  \begin{block}{Матрица парных сравнений}
    Матрицей парных сравнений называется матрица $\boldsymbol{A} = (a_{ij})$, элемент $a_{ij}$ которой показывает, во сколько раз альтернатива $i$ превосходит  $j$ и верно
    \[
      a_{ij} = \nicefrac{1}{a_{ji}}, \quad a_{ij} > 0.
    \]
  \end{block}
  \begin{block}{Согласованная матрица}

\printbibliography[prefixnumbers=FC, heading=none,keyword=fstcat]

    Элементы согласованной матрицы парных сравнений имеют вид [\cite{saaty1984}]
    $$
    a_{i j}=x_{i} / x_{j},
    $$
    где $x_{i}>0-$ компоненты некоторого вектора $\boldsymbol{x}$, называемого вектором абсолютных приоритетов.
  \end{block}
\end{frame}
\begin{frame}{log-чебышёвская аппроксимация}
	На практике, матрицы парных сравнений обычно не согласованы, и поэтому возникает задача приближения матрицы парных сравнений согласованной матрицей.
	
    Одним из способов решения этой задачи (представленный в статье [\cite{krivulin2019}]) является нахождение вектора $x$, который бы минимизировал следующее выражение
    \[
    l_{\infty}\left(\boldsymbol{A}, x x^{-}\right)=\max _{1 \leq i, j \leq n}\left|\log a_{i j}-\log \frac{x_{i}}{x_{j}}\right|, \quad \boldsymbol{x}^{-} = \begin{pmatrix}x_1^{-1}&x_2^{-1} & \dots & x_n^{-1}\end{pmatrix},
    \]
    что сводится к нахождению $x$ такого, что
    \[
      \min _{\boldsymbol{x}>\mathbf{0}} \max _{1 \leq i, j \leq n} \frac{a_{i j} x_{j}}{x_{i}}.
    \]
    Данная задача может быть решена с использованием идемпотентной алгебры и max-алгебры в частности.
\end{frame}
\begin{frame}{Постановка задачи}
  \begin{block}{Max-$\times$-алгебра}
    $\text{Max-}\times$-алгеброй называется алгебра над множеством $\mathbb{R}_{+}=\{x \in \mathbb{R} \mid x \geq 0\}$ с операциями ${(\oplus, \times)}$, где $\oplus$ --- это максимум, а $\times$ --- стандартное умножение.
  \end{block}

	\begin{block}{Однокритериальная задача парного сравнения}
  Соответственно, на языке max-алгебры решение задачи будет выглядеть следующим образом
  \[
    \min _{x>\mathbf{0}} \bigoplus_{1 \leq i, j \leq n} x_{i}^{-1} a_{i j} x_{j},
  \]
  что можно дополнительно переписать как
  \[
    \min _{x>\mathbf{0}} x^{-} A x.
  \]
  \end{block}
\end{frame}
\begin{frame}{Многокритериальная задача парного сравнения}
	\begin{itemize}
		\item 
		Пусть $n$ альтернатив сравниваются по $m$ критериям.
		
		\item
		Матрица $\boldsymbol{A}_k$ --- матрица попарных сравнений по критерию с номером $k = 1, \dots, n$.
		
		\item
		Матрица $\boldsymbol{C} = (c_{ij})$ --- матрица, показывающая, во сколько раз критерий с номером $i$ важнее критерия с номером $j$. 
		
		\item
		Необходимо построить вектор абсолютных приоритетов по матрицам $\boldsymbol{A}_k$ и $\boldsymbol{C}$, чтобы упорядочить альтернативы в соответствии со множеством критериев.
	\end{itemize}
\end{frame}
\begin{frame}{Многокритериальная задача парного сравнения}
		Определим функцию
		$$
		f_{k}(\boldsymbol{x})=d\left(\boldsymbol{A}_{k}, \boldsymbol{x} \boldsymbol{x}^{-}\right)	
		$$ и тогда задача сводится к нахождению
		$$
		\min _{\boldsymbol{x}>\mathbf{0}}\left(f_{1}(\boldsymbol{x}), \ldots, f_{m}(\boldsymbol{x})\right).
		$$
		В терминах тропической математики задача принимает вид
		$$
		\min _{\boldsymbol{x}>\mathbf{0}} \boldsymbol{x}^{-} \boldsymbol{A} \boldsymbol{x},  \text { где }  \boldsymbol{A}=\bigoplus_{1 \leq k \leq m} w_{k} \boldsymbol{A}_{k},
		$$
		где вектор весов $\boldsymbol{w}=\left(w_{k}\right)$ находится как решение задачи
		$$
		\min _{\boldsymbol{w}>0} \boldsymbol{w}^{-} \boldsymbol{C} \boldsymbol{w}.
		$$
\end{frame}


%\begin{frame}{Используемые объекты}
%	Некоторые математические объекты, используемые в решении задачи многокритериального сравнения
%	\begin{block}{Тропический определитель матрицы}
%		\[ 
%		\operatorname{Tr} \boldsymbol{A} \coloneqq \mathrm{tr}\boldsymbol{A} \oplus \dots \oplus \mathrm{tr}\boldsymbol{A}^n
%		\]
%	\end{block}
%	\begin{block}{Спектральный радиус}
%		\[
%		\lambda \coloneqq \mathrm{tr}\boldsymbol{A} \oplus \dots \oplus \mathrm{tr}^{1/n}\boldsymbol{A}^n
%		\]
%	\end{block}
%	\begin{block}{Оператор Клини}
%		Если $\mathrm{Tr} \boldsymbol{A} \leq 1$, то оператор Клини определяется следующим образом
%		\[
%		\boldsymbol{A}^* \coloneqq \boldsymbol{I} \oplus \boldsymbol{A} \oplus \dots \oplus \boldsymbol{A}^{n-1}
%		\]
%	\end{block}
%\end{frame}

\begin{frame}{Мотивация}
	\begin{itemize}
	\item В настоящее время нет общедоступной библиотеки, реализующей данный метод решения многокритериальных задач принятия решений.
	
	\item В качестве языка реализации, из-за развитых библиотек линейной алгебры, а также за потенциально большую скорость работы, был выбран \texttt{C++}.
	
	\item Также было принято решение использовать символьные вычисления, а не численные расчёты. Это обусловлено тем, что используя данный метод решение задачи можно получить в аналитическом виде, а также для большей точности.
	\end{itemize}
\end{frame}

\begin{frame}{Мотивация}
  Выбор используемых технологий обусловлен следующим:
	\begin{itemize}
    \item Встраиваемость: например, библиотеку на \texttt{C++} можно использовать в языках \texttt{Python}, \texttt{R}, \texttt{Javascript} (через \texttt{NodeJS}) и многих других.
    \item Доступность библиотек с открытым исходным кодом позволяет использовать библиотеку на бóльшем числе процессорных архитектур и операционных систем.
    \item Данное решение более специализировано за счёт чего можно достичь бóльшей производительности и меньшего веса программы, по сравнению с написанием программы для существующих систем компьютерной алгебры.
	\end{itemize}
\end{frame}

\begin{frame}
	\begin{block}{Цель работы}
		Написать библиотеку и программу на языке \texttt{C++}, символьно решающую задачу принятия решений методом log-чебышёвской аппроксимации в max-алгебре.
	\end{block}
	
	\begin{block}{Результаты}
		\begin{enumerate}
		  \item Были релизованы max-+ и max-$\times$-алгебры и добавлена их поддержка в библиотеку линейной алгебры \texttt{Eigen} (версия 3.3.8).
      \item Все вычисление проводятся символьно (при помощи библиотеки \texttt{GiNaC} версии 1.8.0).
			\item Был реализован набор функций (тропический определитель, спектральный радиус, оператор Клини), используемых в алгоритме решения задачи.
			\item Многокритериальная задача принятия решений может быть решена полностью при помощи программы.
		\end{enumerate}
	\end{block}
\end{frame}

\begin{frame}
Библиотека представляет собой
		\begin{enumerate}
			\item Класс \texttt{MaxAlgebra} реализующий операции в max-алгебре.
			\item Две специализации класса реализующие операции в max-+ и max-$\times$-алгебре.
			\item Расширение для библиотеки \texttt{Eigen}, позволяющее использовать \texttt{MaxAlgebra} в качестве элементов матрицы.
			\item Набор функций, участвующих в алгоритме решения задачи: нахождение тропического определителя, спектрального радиуса, оператор Клини и т. д.
		\end{enumerate}

\end{frame}

\begin{frame}{Дальнейшие планы}
	\begin{enumerate}
		\item Реализовать возможность получения полного решения многокритериальной задачи принятия решений в полностью автоматическом режиме.
		\item Протестировать скорость работы программы относительно существующих решений.
	\end{enumerate}
\end{frame}

\defverbatim[colored]\lstI{
\begin{lstlisting}[language=C++,basicstyle=\scriptsize\ttfamily,keywordstyle=\color{red}]
MatrixDn A1(3, 3);
MatrixDn A2(3, 3);
A1 << ex(sin(1)), 3, ex(1) / 3,
       ex(1) / 3, 1, 1,
               3, 1, 1;
A2 <<   0, ex(1) / 3, 5,
        3,         1, 7,
ex(1) / 5, ex(1) / 7, 1;
\end{lstlisting}
}
\defverbatim[colored]\lstII{
\begin{lstlisting}[language=bash,basicstyle=\scriptsize\ttfamily,keywordstyle=\color{red}]
A1 + A2 =               A1 * A2 =
  \sin(1) 3 5             9  3 21
  3       1 7             3  1  7
  3       1 1             3  1 15
\end{lstlisting}
}
\defverbatim[colored]\lstIII{
\begin{lstlisting}[language=bash,basicstyle=\footnotesize\ttfamily,keywordstyle=\color{red}]
  MatrixDn C(4, 4);
  C <<  8, 3, 10, 6,
        7, 2,  1, 4,
       10, 8,  2, 2,
        4, 7,  6, 1;
  spectral_radius(C);
\end{lstlisting}
}
\defverbatim[colored]\lstIV{
\begin{lstlisting}[language=bash,basicstyle=\tiny\ttfamily,keywordstyle=\color{red}]
m^1=          m^2=                m^3=                       m^4=
8  3 10  6   100  80  80  48     800  640 1000  600         10000  8000  8000  4800
7  2  1  4    56  28  70  42     700  560  560  336          5600  4480  7000  4200
10 8  2  2    80  30 100  60    1000  800  800  480          8000  6400 10000  6000
4  7  6  1    60  48  40  28     480  320  600  360          6000  4800  4800  2880
              tr(m^2)^(1/2) = 10  tr(m^3)^(1/3) = 800^(1/3)  tr(m^4)^(1/4) = 10

Spectral radius of C = 10
\end{lstlisting}
}
\defverbatim[colored]\lstV{
	\begin{lstlisting}[language=bash,basicstyle=\scriptsize\ttfamily,keywordstyle=\color{red}]
auto lambda = spectral_radius(C);
	\end{lstlisting}
}
\defverbatim[colored]\lstVI{
	\begin{lstlisting}[language=bash,basicstyle=\scriptsize\ttfamily,keywordstyle=\color{red}]
MatrixDn Calmoststar = 1 / lambda * C;
auto w = cleany(Calmoststar);
	\end{lstlisting}
}
\defverbatim[colored]\lstVII{
	\begin{lstlisting}[language=c++,basicstyle=\scriptsize\ttfamily]
auto delta = (MatrixDn::Ones(1, 6) * 
                                 w * 
      MatrixDn::Ones(6, 1)).value();
	\end{lstlisting}
}
\defverbatim[colored]\lstVIII{
	\begin{lstlisting}[language=c++,basicstyle=\footnotesize\ttfamily]
MatrixDn W1m = (1 / delta * MatrixDn::Ones(6, 6)) + Calmoststar;
auto w1 = cleany(W1m);
	\end{lstlisting}
}
\defverbatim[colored]\lstIX{
	\begin{lstlisting}[language=c++,basicstyle=\footnotesize\ttfamily]
auto worst_vector =
(w.col(0) / w.col(0).maxCoeff()) + (w.col(1) / w.col(1).maxCoeff());
	\end{lstlisting}
}
\defverbatim[colored]\lstX{
	\begin{lstlisting}[language=c++,basicstyle=\footnotesize\ttfamily]
auto B1 =
(worst_vector(0) * A1 + worst_vector(1) * A2 + worst_vector(2) * A3 +
worst_vector(3) * A4 + worst_vector(4) * A5 + worst_vector(5) * A6)
.eval();

auto nu = spectral_radius(B1);

MatrixDn almost_x1 = 1 / nu * B1;
auto x1 = cleany(almost_x1);
	\end{lstlisting}
}
\defverbatim[colored]\lstXI{
	\begin{lstlisting}[language=c++,basicstyle=\footnotesize\ttfamily]
auto delta1 = (MatrixDn::Ones(1, 3) * x1 * MatrixDn::Ones(3, 1)).value();
MatrixDn almost_x1new = (delta1 * MatrixDn::Ones(3, 3)) + almost_x1;
auto x1new = cleany(almost_x1new);
	\end{lstlisting}
}
\defverbatim[colored]\lstXII{
	\begin{lstlisting}[language=c++,basicstyle=\footnotesize\ttfamily]
MatrixDn P(6, 2);
P << w.col(0), w.col(1);
	\end{lstlisting}
}
\defverbatim[colored]\lstXIII{
	\begin{lstlisting}[language=c++,basicstyle=\footnotesize\ttfamily]
auto [k, l] = best_diff_vector_coefficients(P);
	\end{lstlisting}
}
\defverbatim[colored]\lstXIV{
	\begin{lstlisting}[language=c++,basicstyle=\footnotesize\ttfamily]
MatrixDn Plk_inverse = MatrixDn::Zero(P.rows(), P.cols());
Plk_inverse(l, k) = 1 / P(l, k);
		
auto w2 = P * (MatrixDn::Identity(P.cols(), P.cols()) +
Plk_inverse.transpose() * P);
	\end{lstlisting}
}

\defverbatim[colored]\lstVr{
\begin{lstlisting}[language=c++,basicstyle=\footnotesize\ttfamily]
lambda = (360/7)^(1/5)
\end{lstlisting}
}
\defverbatim[colored]\lstVIr{
\begin{lstlisting}[language=c++,basicstyle=\scriptsize\ttfamily]
w = 
[1, 1/20*(360/7)^(2/5), 1/60*(360/7)^(3/5), 1/4*(360/7)^(1/5), 7/60*(360/7)^(4/5), 
1/20*(360/7)^(2/5)]
[7/30*(360/7)^(3/5), 1, 1/5*(360/7)^(1/5), 7/120*(360/7)^(4/5), 7/5*(360/7)^(2/5), 
3/5]
[7/6*(360/7)^(2/5), 7/120*(360/7)^(4/5), 1, 7/24*(360/7)^(3/5), 7*(360/7)^(1/5), 
7/120*(360/7)^(4/5)]
[7/90*(360/7)^(4/5), 1/5*(360/7)^(1/5), 1/15*(360/7)^(2/5), 1, 7/15*(360/7)^(3/5), 
1/5*(360/7)^(1/5)]
[1/6*(360/7)^(1/5), 1/120*(360/7)^(3/5), 1/360*(360/7)^(4/5), 1/24*(360/7)^(2/5), 1, 
1/120*(360/7)^(3/5)]
[7/18*(360/7)^(3/5), 1, 1/3*(360/7)^(1/5), 7/72*(360/7)^(4/5), 7/3*(360/7)^(2/5), 
1]
\end{lstlisting}
}
\defverbatim[colored]\lstVIIr{
\begin{lstlisting}[language=c++,basicstyle=\footnotesize\ttfamily]
delta = 7*(360/7)^(1/5)
delta^-1 = 1/360*(360/7)^(4/5)
\end{lstlisting}
}
\defverbatim[colored]\lstVIIIr{
	\begin{lstlisting}[language=c++,basicstyle=\scriptsize\ttfamily]
w1 = 
[1, 1/20*(360/7)^(2/5), 1/60*(360/7)^(3/5), 1/4*(360/7)^(1/5), 7/60*(360/7)^(4/5), 
1/20*(360/7)^(2/5)]
[7/30*(360/7)^(3/5), 1, 1/5*(360/7)^(1/5), 7/120*(360/7)^(4/5), 7/5*(360/7)^(2/5), 
3/5]
[7/6*(360/7)^(2/5), 7/120*(360/7)^(4/5), 1, 7/24*(360/7)^(3/5), 7*(360/7)^(1/5), 
7/120*(360/7)^(4/5)]
[7/90*(360/7)^(4/5), 1/5*(360/7)^(1/5), 1/15*(360/7)^(2/5), 1, 7/15*(360/7)^(3/5), 
1/5*(360/7)^(1/5)]
[1/6*(360/7)^(1/5), 1/120*(360/7)^(3/5), 1/360*(360/7)^(4/5), 1/24*(360/7)^(2/5), 1, 
1/120*(360/7)^(3/5)]
[7/18*(360/7)^(3/5), 1, 1/3*(360/7)^(1/5), 7/72*(360/7)^(4/5), 7/3*(360/7)^(2/5), 
1]
	
worst_vector = 
[1/60*(360/7)^(3/5)]
[1/3*(360/7)^(1/5)]
[1]
[1/15*(360/7)^(2/5)]
[1/360*(360/7)^(4/5)]
[1/3*(360/7)^(1/5)]
	
B1 = 
[1, 7/3*(360/7)^(1/5), 3*(360/7)^(1/5)]
[7, 1, 7/3*(360/7)^(1/5)]
[9, 7, 1]
nu = (360/7)^(1/10)*sqrt(27)
x1 = 
[1, 7/9, 1/9*(360/7)^(1/10)*sqrt(27)]
[49/9720*(360/7)^(9/10)*sqrt(27), 1, 7/9]
[7/1080*(360/7)^(9/10)*sqrt(27), 49/9720*(360/7)^(9/10)*sqrt(27), 1]
x1new = 
[7/120*(360/7)^(4/5), 7/120*(360/7)^(4/5), 7/120*(360/7)^(4/5)]
[7/120*(360/7)^(4/5), 7/120*(360/7)^(4/5), 7/120*(360/7)^(4/5)]
[7/120*(360/7)^(4/5), 7/120*(360/7)^(4/5), 7/120*(360/7)^(4/5)]
\end{lstlisting}
}
\defverbatim[colored]\lstIXr{
	\begin{lstlisting}[language=c++,basicstyle=\footnotesize\ttfamily]
P = [1, 1/20*(360/7)^(2/5)]
[7/30*(360/7)^(3/5), 1]
[7/6*(360/7)^(2/5), 7/120*(360/7)^(4/5)]
[7/90*(360/7)^(4/5), 1/5*(360/7)^(1/5)]
[1/6*(360/7)^(1/5), 1/120*(360/7)^(3/5)]
[7/18*(360/7)^(3/5), 1]
\end{lstlisting}
}
\defverbatim[colored]\lstXr{
	\begin{lstlisting}[language=c++,basicstyle=\footnotesize\ttfamily]
k = 0, l = 4
\end{lstlisting}
}
\defverbatim[colored]\lstXIr{
	\begin{lstlisting}[language=c++,basicstyle=\footnotesize\ttfamily]
w2 = 
[1, 1/20*(360/7)^(2/5)]
[7/30*(360/7)^(3/5), 1]
[7/6*(360/7)^(2/5), 7/120*(360/7)^(4/5)]
[7/90*(360/7)^(4/5), 1/5*(360/7)^(1/5)]
[1/6*(360/7)^(1/5), 1/120*(360/7)^(3/5)]
[7/18*(360/7)^(3/5), 1]
\end{lstlisting}
}
\defverbatim[colored]\lstXIIr{

}
\defverbatim[colored]\lstXIIIr{

}
%
%\begin{frame}{Пример. Арифметика}
%\begin{block}{Код}
%\lstI
%\end{block}
%\begin{block}{Результат}
%\lstII
%\end{block}
%\end{frame}
%\begin{frame}{Пример. Спектральный радиус}
%\begin{block}{Код}
%\lstIII
%\end{block}
%\begin{block}{Результат}
%\lstIV
%\end{block}
%\end{frame}
%\begin{frame}{Пример. Элементы решения задачи}
%Дальше будут представлены некоторые части решения реальной задачи принятия решений.\\
%\textbf{Условие}
%$$
%\footnotesize\begin{array}{l}
%	C=\left(\begin{array}{cccccc}
%		1 & 1 / 4 & 1 / 5 & 1 / 4 & 6 & 1 / 6 \\
%		4 & 1 & 1 / 3 & 3 & 6 & 1 / 2 \\
%		5 & 3 & 1 & 4 & 7 & 3 \\
%		4 & 1 / 3 & 1 / 4 & 1 & 5 & 1 / 5 \\
%		1 / 6 & 1 / 6 & 1 / 7 & 1 / 5 & 1 & 1 / 7 \\
%		6 & 2 & 1 / 3 & 5 & 7 & 1
%	\end{array}\right) \\
%	A_{1}=\left(\begin{array}{ccc}
%		1 & 5 & 8 \\
%		1 / 5 & 1 & 5 \\
%		1 / 8 & 1 / 5 & 1
%	\end{array}\right), \quad A_{2}=\left(\begin{array}{ccc}
%		1 & 7 & 9 \\
%		1 / 7 & 1 & 7 \\
%		1 / 9 & 1 / 7 & 1
%	\end{array}\right), \quad A_{3}=\left(\begin{array}{ccc}
%		1 & 1 / 7 & 1 / 9 \\
%		7 & 1 & 1 / 7 \\
%		9 & 7 & 1
%	\end{array}\right), \\
%	A_{4}=\left(\begin{array}{ccc}
%		1 & 3 & 5 \\
%		1 / 3 & 1 & 4 \\
%		1 / 5 & 1 / 4 & 1
%	\end{array}\right), \quad A_{5}=\left(\begin{array}{ccc}
%		1 & 3 & 5 \\
%		1 / 3 & 1 & 4 \\
%		1 / 5 & 1 / 4 & 1
%	\end{array}\right), \quad A_{6}=\left(\begin{array}{ccc}
%		1 & 7 & 9 \\
%		1 / 7 & 1 & 7 \\
%		1 / 9 & 1 / 7 & 1
%	\end{array}\right)
%\end{array}
%$$
%\end{frame}
%\begin{frame}{Вычисление спектрального радиуса}
%
%	\begin{block}{Код}
%	\lstV
%	\end{block}
%\begin{block}{Результат}
%	\lstVr
%\end{block}
%\end{frame}
%\begin{frame}{Вычисление $(\lambda^{-1}C)^*$}
%\begin{block}{Код}
%\lstVI
%\end{block}
%\end{frame}
%\begin{frame}{Вычисление $\delta$}
%	\begin{block}{Код}
%		\lstVII
%	\end{block}
%	\begin{block}{Результат}
%		\lstVIIr
%	\end{block}
%\end{frame}
%


%\begin{frame}{Литература}
%	\printbibliography
%\end{frame}


\end{document}
